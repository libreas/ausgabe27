\documentclass[a4paper,
fontsize=11pt,
%headings=small,
oneside,
numbers=noperiodatend,
parskip=half-,
bibliography=totoc,
final
]{scrartcl}

\usepackage{synttree}
\usepackage{graphicx}
\setkeys{Gin}{width=.4\textwidth} %default pics size

\graphicspath{{./plots/}}
\usepackage[ngerman]{babel}
\usepackage[T1]{fontenc}
%\usepackage{amsmath}
\usepackage[utf8x]{inputenc}
\usepackage [hyphens]{url}
\usepackage{booktabs} 
\usepackage[left=2.4cm,right=2.4cm,top=2.3cm,bottom=2cm,includeheadfoot]{geometry}
\usepackage{eurosym}
\usepackage{multirow}
\usepackage[ngerman]{varioref}
\setcapindent{1em}
\renewcommand{\labelitemi}{--}
\usepackage{paralist}
\usepackage{pdfpages}
\usepackage{lscape}
\usepackage{float}
\usepackage{acronym}
\usepackage{eurosym}
\usepackage[babel]{csquotes}
\usepackage{longtable,lscape}
\usepackage{mathpazo}
\usepackage[flushmargin,ragged]{footmisc} % left align footnote

\usepackage{listings}

\urlstyle{same}  % don't use monospace font for urls

\usepackage[fleqn]{amsmath}

%adjust fontsize for part

\usepackage{sectsty}
\partfont{\large}

%Das BibTeX-Zeichen mit \BibTeX setzen:
\def\symbol#1{\char #1\relax}
\def\bsl{{\tt\symbol{'134}}}
\def\BibTeX{{\rm B\kern-.05em{\sc i\kern-.025em b}\kern-.08em
    T\kern-.1667em\lower.7ex\hbox{E}\kern-.125emX}}

\usepackage{fancyhdr}
\fancyhf{}
\pagestyle{fancyplain}
\fancyhead[R]{\thepage}

%meta
%meta

\fancyhead[L]{F. Hüppi \\ %author
LIBREAS. Library Ideas, 27 (2015). % journal, issue, volume.
\href{http://nbn-resolving.de/urn:nbn:de:kobv:11-100229817
}{urn:nbn:de:kobv:11-100229817}} % urn
\fancyhead[R]{\thepage} %page number
\fancyfoot[L] {\textit{Creative Commons BY 3.0}} %licence
\fancyfoot[R] {\textit{ISSN: 1860-7950}}

\title{\LARGE{Interviews und Beobachtungen als Methodenkombination. Auch für
Bibliotheken?}} %title %title
\author{Felix Hüppi} %author

\setcounter{page}{3}

\usepackage[colorlinks, linkcolor=black,citecolor=black, urlcolor=blue,
breaklinks= true]{hyperref}

\date{}
\begin{document}

\maketitle
\thispagestyle{fancyplain} 

%abstracts

%body
\section*{Einleitung}\label{einleitung}

Im letzten Jahr entstand meine Masterarbeit zum Thema Lernraum in
Schweizer Hochschulbibliotheken (Hüppi 2014). Neben der Evaluation des
Lernraumangebots war ein Ziel der Arbeit, eine Kombination von
Experteninterviews mit einer Beobachtung einzusetzen. Ich untersuchte in
der Arbeit, ob dieser Methodenmix einen Mehrwert besitzt oder ob die
Methoden sinnvoller einzeln eingesetzt werden sollten. Daraus ergaben
sich Erkenntnisse, die für die Praxis relevant sein können. Im folgenden
werden der methodische Ablauf und die Erkenntnisse daraus kurz
wiedergeben. Dann werden mögliche Umsetzungsformen dieser Erkenntnisse
in die Praxis diskutiert. Es ist es wert, sowohl die beiden Methoden
einzeln als auch besonders in deren Kombination zu betrachten\textbf{.}
Gerade dieser Teil fehlt den wissenschaftlich angelegten Arbeiten der
Hochschulen oft, da sie strengen Anforderungen unterliegen, während in
der Praxis ein einfacher und pragmatischer Ansatz gefordert ist.

Dann wird sich dieses Essay mit der praktischen Umsetzung der Methoden
und des Methodenmix befassen, da diese oft nur kurz abgehandelt wird,
für Bibliotheken aber zentral ist. Mein Hintergrund als Bibliothekar in
einer Öffentlichen Bibliothek wird diese Betrachtungen etwas
beeinflussen, ich bemühe mich aber, einen möglichst offenen Blick zu
behalten.

\section*{Ausgangslage}\label{ausgangslage}

Die Masterarbeit befasste sich mit dem Lernraum in Schweizer
Hochschulbibliotheken. Einleitend wurde die Theorie zum Thema
Lernverhalten, Lernen und Lernraum betrachtet. Anschliessend folgte eine
empirische Untersuchung zu diesem Thema. Dafür wurden fünf Personen aus
Schweizer Hochschulbibliotheken, die für das Thema Lernraum zuständig
sind, interviewt. Anschliessend wurde eine Beobachtung von Studierenden
beim Lernen in einer Bibliothek durchgeführt. Ein zentrales Anliegen der
Masterarbeit war in Berücksichtigung einschlägiger Literatur die
Evaluation dieses spezifischen Methodenmixes, einer Kombination von
Experteninterviews und Beobachtungen. Zunächst wurden Experteninterviews
durchgeführt und nutzte danach die Beobachtung, um die Ergebnisse der
Interviews zu überprüfen, zu widerlegen oder zu verdeutlichen. Die
Triangulation, also der Vergleich und die zusammenfassenden
Schlussfolgerungen erfolgten daraus. Anschliessend wurden die
gesammelten Erkenntnisse am Beispiel der neu zu erstellenden
Campusbibliothek
Muttenz der
Fachhochschule Nordwestschweiz in die Praxis umgesetzt und Vorschläge
für diese Bibliothek ausgearbeitet.

\section*{Interviews}\label{interviews}

Als Grundlage für die Experteninterviews diente ein teilstrukturierter
Leitfaden, der bei allen fünf Interviews verwendet wurde. Dies
entspricht der Expertenbefragung von Atteslander (2008, S. 132) oder dem
problemzentrierten Interview gemäss Mayring (2002, S. 67). Die Inhalte
der Leitfadengespräche wurden durch Notizen während des Interviews,
durch die Anfertigung eines Gedächtnisprotokolls nach dem Gespräch und
durch Tonbandaufzeichnungen festgehalten. Die Auswertung wurde nach
qualitativen Kriterien durchgeführt, sinnvolle quantitative Auswertungen
liessen sich bei einer Stichprobe von fünf Bibliotheken nicht machen.
Die Ergebnisse wertete ich deshalb mit einer qualitativen Inhaltsanalyse
gemäss Mayring aus (2002, S. 114ff.). Diese Anzahl von fünf Interviews
war willkürlich gewählt und lässt sich nicht gut begründen. Es sind zu
wenige Interviews um repräsentative Aussagen zur Thematik zu machen. Es
bleiben Stichproben, wozu vermutlich auch weniger Interviews gereicht
hätten. Eine umfangreiche Datenlage ist grundsätzlich aussagekräftiger
aber im Rahmen einer Masterarbeit darf der Aufwand pro Interview nicht
unterschätzt werden.

Ein Problem bei Interviews als Methode wird immer die Repräsentativität
bleiben, wie es auch beispielsweise Atteslander thematisiert (2008, S.
61). Die Befragung von fünf Personen an Hochschulbibliotheken, die in
den letzten Jahren um- oder neugebaut wurden, ergibt keine Datenbasis,
die solide genug ist, um auf die gesamte Hochschullandschaft der Schweiz
rückschliessen zu können. Die Ergebnisse bleiben deshalb nur Indikatoren
dafür, wie die tatsächlichen Verhältnisse sein könnten. Vogel und Woisch
(2013) führten eine repräsentative Studie in Deutschland zum Thema
\enquote{Orte des Selbststudiums} durch und befragten dafür 34`886
Studierende, was zu repräsentativen Ergebnissen führte.

Ein zweites Problem ist die Zentralität der Interviewpartner, also ihre
persönliche Betroffenheit (Atteslander 2008, S. 61). Bei Fragen zum
Lernverhalten der Studierenden und Fragen zu hochschulweiten Strategien
sind sie nur indirekt betroffen und können nur Auskünfte aus zweiter
Hand geben. Durch die anschliessende Beobachtung konnte dieses Problem
der Zentralität etwas entschärft werden, da dabei persönlich Betroffene,
also Lernende, im Fokus standen.

Ein Problem bei Experteninterviews ist weiterhin oft das Fehlen der
unmittelbaren Nähe zum untersuchten Gegenstand. Auch bei den Interviews
für die Masterarbeit war dies der Fall, da die befragten Personen nicht
selbst in der Bibliothek lernen und deshalb keine unmittelbaren
Informationen weitergeben konnten. So fiel beispielsweise mehreren
Befragten auf, wie stark sich Studierende an die Zeiten der Vorlesungen
hielten und immer zur gleichen Zeit Pause machten, auch beim
individuellen Lernen. Eine Erklärung dazu konnten sie aber nicht
liefern. Deshalb war die Kombination mit einer zweiten Methode sinnvoll.

\section*{Beobachtung}\label{beobachtung}

Für die Beobachtung diente ein Beobachtungsleitfaden und eine Bestimmung
der Beobachtungsdimensionen. Diese Dimensionen ergaben sich aus der
Auswertung der Interviews, da die Beobachtung zur Validierung der
Ergebnisse der Interviews diente. Der Verfasser erstellte entsprechende
Hypothesen und operationalisierte diese, woraus sich dann die
Beobachtungseinheiten ergaben. Aus Kapazitätsgründen wurde die
Beobachtung auf wenige Dimensionen beschränkt und solche Dimensionen
gewählt, die sich gut beobachten liessen, ohne dabei die Studierenden zu
stören. Die Beobachtungseinheiten waren in verschiedenen
Beobachtungsschemata operationalisiert und festgehalten, welche dann bei
der Beobachtung eingesetzt wurden. Die untenstehende Tabelle (Tabelle \vref{tab:1})
war eines dieser Beobachtungsschemata. Die persönliche Teilnahme des
Autors beschränkte sich bei dieser Beobachtung auf blosse Anwesenheit
ohne Interaktion mit den Studierenden, da vor allem beim stillen Lernen
eine Teilnahme sowieso kaum möglich ist.

\begin{tiny}
\begin{longtable}[c]{p{0.7cm}p{0.7cm}p{0.7cm}p{0.7cm}p{0.7cm}p{0.7cm}p{0.7cm}p{0.7cm}p{0.7cm}p{0.7cm}p{0.7cm}p{0.7cm}p{0.7cm}p{0.7cm}}
\toprule
Zeit / Verhalten & Computer nutzen & Lernen mit Karten & Mobil telefon
nutzen & Schreiben & Lesen markieren & Multi tasking & Sprechen & Pausen
Schlafen & Essen trinken & Theke DL nutzen & Zeitung lesen & Umgang mit
Lärm\tabularnewline
\midrule
\endhead
08.00-08.15 & & & & & & & & & & & &\tabularnewline
08.15-08.30 & & & & & & & & & & & &\tabularnewline
08.30-08.45 & & & & & & & & & & & &\tabularnewline
08.45-09.00 & & & & & & & & & & & &\tabularnewline
09.00-09.15 & & & & & & & & & & & &\tabularnewline
09.15-09.30 & & & & & & & & & & & &\tabularnewline
09.30-09.45 & & & & & & & & & & & &\tabularnewline
\ldots{} & & & & & & & & & & & &\tabularnewline
- & & & & & & & & & & & &\tabularnewline
\ldots{} & & & & & & & & & & & &\tabularnewline
\bottomrule
\caption{Beobachtungsschema zum Lernverhalten von Studierenden
}
\label{tab:1}
\end{longtable}
\end{tiny}

Die Auswertung wurde, wie bei den Interviews, durch eine qualitative
Inhaltsanalyse vorgenommen. Die Ergebnisse wurden mit den Erkenntnissen
aus der Literatur und den gewonnenen Ergebnissen der Interviews
verglichen. So konnten Gemeinsamkeiten und Unterschiede festgestellt und
die Ergebnisse der beiden anderen Methoden validiert werden.

Einschränkend ist zu sagen, dass das studentische Lernen durch das
Semester bestimmt wird. Vor den Prüfungen wird besonders intensiv
gelernt, in der übrigen Zeit eher wenig. Die Beobachtung fand im Juni
statt, zu einer Zeit, in der die Prüfungsphase lag. Die angetroffene
Situation gab deshalb nicht das vollständige Bild des Lernverhaltens
wieder, sondern präsentiert hauptsächlich das prüfungsvorbereitende
Lernen.

Ein grundsätzliches Problem bei Beobachtungen ist die Selektivität der
Wahrnehmung (Atteslander 2008, S. 95). Der Beobachter hat nur eine
begrenzte Wahrnehmungsspanne und ist durch die Vorgaben des
Beobachtungsschemas auf gewisse Dinge fixiert, sodass er andere
übersieht. Daneben ergeben sich aus der Rolle des Beobachters ebenfalls
Einschränkungen, da persönliche Aspekte immer einen Einfluss haben. Bei
dieser Beobachtung hatte der Autor schon relativ klare Vorstellungen zum
Thema, die zu bestätigen oder zu widerlegen er versuchte. Dadurch
entgingen vermutlich lernrelevante Verhaltensweisen, die nicht diesen
Vorstellungen entsprachen. Eine gute Übersicht zu den Problemen, die mit
Beobachtungen einhergehen, gibt Schöne (2003, Kapitel 3.4.2
Schwierigkeiten bei der Feldarbeit).

Der Vorteil der Beobachtung liegt jedoch darin, dass keine Verzerrung
durch soziale Wunschvorstellung oder die Erwartungshaltung des
Forschungssubjekts entstehen. Durch die Beobachtung können die
Ergebnisse der Interviews direkter kontrolliert werden als durch eine
weitere Gesprächssituation. Sie eignet sich deshalb für die Kombination
mit einer Gesprächsmethode. Die Beobachtung ist aber eine anspruchsvolle
Methode und weniger verbreitet als Gesprächsmethoden, was bei der
Anwendung jeweils berücksichtigt werden muss.

Eine Beobachtung durchzuführen ist eine komplexe Angelegenheit. Sie
braucht eine sehr gute Vorbereitung und ein hohes Mass an Konzentration
bei der Durchführung. Die anschliessende Auswertung ist sehr aufwändig.
Damit die Beobachtung gute Ergebnisse bringt, würde es sich lohnen, die
Methode in informationswissenschaftlichen Studienrichtungen besser zu
schulen. Sie führt bis jetzt ein Schattendasein, die in verschiedensten
Formen geübt und gelehrt wird.

\section*{Ergebnisse und Vorteile bei der Verwendung mehrerer
Methoden}\label{ergebnisse-und-vorteile-bei-der-verwendung-mehrerer-methoden}

Beim Einsatz von Forschungsmethoden wird im Voraus über Aufwand und
Ertrag nachgedacht. Bei der Verwendung von mehreren Methoden sind diese
Überlegungen noch wichtiger, da der Aufwand ungleich grösser ist. Es
müssen sowohl der Umgang mit zwei oder mehreren Methoden überlegt, als
auch anschliessend die Ergebnisse der verschiedenen Methoden miteinander
verglichen und daraus Schlüsse gezogen werden.

Die Vorteile bei der Verwendung von mehreren Methoden beschreibt Mayring
(2002). Er bezeichnet den Einsatz von mehreren Methoden als
Triangulation, ein passendes Bild, welches das Ziel zeigt, ein möglichst
genaues Ergebnis einer Situation zu erhalten.

Durch die gleichzeitige Anwendung der beiden erwähnten Methoden ergibt
sich eine Sicht aus verschiedenen Perspektiven. Experten haben meist
eine Aussensicht, während durch die Beobachtung eine Innensicht des
Anwenders oder Nutzers gezeigt werden kann.

Die gewählte Methodenkombination kann gut für die Evaluierung aktueller
Angebote und deren Nutzung verwendet werden. Weniger sinnvoll ist sie
bei der Voraussage von Ereignissen, dem Versuch, zukünftige Trends oder
zusätzliche Bedürfnisse zu erfahren oder die Meinungen von Personen
einzuholen. Die Beobachtung ist für solche Aussagen nicht die geeignete
Methode.

Alternativen wären Interviews mit den Lernenden selbst anstatt mit
Experten. Bei der Lernraumthematik wurden diese in einer anderen
Masterarbeit durchgeführt. Weitere Alternativen führt beispielsweise die
Swiss Academy for Development (2015) auf.

\section*{Praktische Umsetzung}\label{praktische-umsetzung}

Mayring nennt sechs allgemeine Gütekriterien qualitativer Forschung:
Verfahrensdokumentation, argumentative Interpretationsabsicherung,
Regelgeleitetheit, Nähe zum Gegenstand, kommunikative Validierung und
Triangulation (2002, S. 144ff.). Für wissenschaftliches Arbeiten sind
diese Kriterien sicher wichtig, für den Einsatz der Methoden im
Bibliotheksalltag ist damit die Latte aber eher zu hoch gelegt. Es ist
auch für Ergebnisse in der Praxis wichtig, dass sie nachvollziehbar und
transparent sind. Sie müssen aber sicher nicht solch hohen Anforderungen
genügen, dafür gibt es meist keinen Bedarf.

Viel wichtiger ist es, Aufwand und Ertrag gut abzuwägen. Um den
kombinierten Einsatz von Befragungen und Beobachtungen zu rechtfertigen,
sollten wichtige oder weitreichende Fragen geklärt werden; die Qualität
der Vorgehensweise ist in jedem Fall wichtig. Möglicherweise sollen die
Ergebnisse externe Geldgeber überzeugen oder den Bibliotheksträger zu
zusätzlicher Finanzierung anregen. Dafür braucht es jeweils überzeugende
Argumente.

Es können aber auch für kleinere Fragestellungen Methoden der
empirischen Sozialforschung eingesetzt werden, dann allerdings auf
einfachere Art und Weise. Beispielsweise kann bei Fragen zu Mobiliar und
Einrichtung relativ unkompliziert eine Beobachtung des Kundenverhaltens
durchgeführt und auf einem einfachen Beobachtungsschema festgehalten
werden. Trotzdem soll damit nicht übertrieben und solche Methoden als
Ersatz für Eigeninitiative verwendet werden. Würde jede Handlung
untersucht und evaluiert werden , wäre die Bibliothek sicher dadurch
schwerfällig und unflexibel.

Neben Aufwand und Ertrag findet der Autor die Wahl der Methode wichtig.
Es gibt einen Aphorismus von Abraham Maslow: \emph{If you only have a
hammer, you tend to see every problem as a nail}. Etwas weniger prägnant
drückt sich Mayring (2002, S. 149) aus, dem es wichtig ist, die
Gegenstände nicht der Methode unterzuordnen, sondern die richtige
Methode für den zu untersuchenden Sachverhalt zu finden.

In den Bibliotheken besteht die Tendenz, als Instrument eine Umfrage zu
wählen und dann zu definieren, was man wissen möchte. Dies bietet den
Vorteil, dass in vielen Bibliotheken ab einer gewissen Grösse dieses
Instrument gut beherrscht wird. Trotzdem ergeben sich dabei Probleme wie
die soziale Erwünschtheit von Antworten, Selbstselektion oder Einfluss
des Interviewers.

Um Methoden der empirischen Sozialwissenschaft einzusetzen, sind
geschulte Mitarbeitende nötig. Nur dann ist effizientes Arbeiten
möglich, da es eine gewisse Vertrautheit mit den Methoden braucht und
meist erst nach mehrmaligem Einsatz gut damit umgegangen werden kann.
Alle Methoden sind aber grundsätzlich gut lernbar.

Als Einsatzgebiet für Beobachtungen kommen alle Themen in Frage, bei
denen Bibliotheksnutzende in der Bibliothek selbst agieren. Das sind
beispielsweise Fragen zum Bestand, Räumlichkeiten, Dienstleistungs- und
Serviceangebot oder Öffnungszeiten.

Die hier propagierte Methodenkombination kann gut für die Evaluierung
aktueller Angebote und deren Nutzung angewendet werden. Weniger sinnvoll
ist sie bei der Voraussage von Ereignissen, dem Versuch, zukünftige
Trends oder zusätzliche Bedürfnisse zu erfahren oder die Meinungen von
Personen einzuholen.

Ein Vorteil von empirischen sozialwissenschaftlichen Methoden liegt
darin, dass sie die Bibliotheksnutzenden miteinbeziehen, was meistens
geschätzt wird. Die Bibliotheksbesucher fühlen sich ernst genommen. So
ist der Einsatz zudem öffentlichkeitswirksam.

\section*{Fazit}\label{fazit}

Für eine Kombination von zwei Methoden gibt es verschiedene sinnvolle
Einsatzgebiete und es können damit gute Ergebnisse erzielt werden. Der
Methodenmix kann gut für die Evaluierung aktueller Angebote und deren
Nutzung angewendet werden. Der Einsatz im Bibliotheksalltag ist
allerdings relativ aufwändig und braucht Mitarbeitende, die darin
geschult sind. Es ergeben sich dadurch aber aussagekräftige Ergebnisse,
die auch für schwierige Entscheide eine gute Grundlage sein können.

Dieses Essay soll grundsätzlich ein Plädoyer für die Anwendung von
Methoden der empirischen Sozialforschung in der Bibliothek sein. Auch
kleine und einfache Methodeneinsätze können spannende Erkenntnisse
bringen und der Aufwand hält sich mit etwas Übung in Grenzen.

\section*{Literatur}\label{literatur}

Atteslander, Peter (2008): \emph{Methoden der empirischen
Sozialforschung}. 12. Aufl. Berlin: Erich Schmidt Verlag.

Hüppi, Felix (2014): \emph{Lernraum Bibliothek: Theorie und Schweizer
Praxis mit Umsetzungsbeispiel für die neue Campusbibliothek Muttenz der
FHNW.} Masterarbeit. Chur: HTW Chur.

Mayring, Philipp (2002): \emph{Einführung in die qualitative
Sozialforschung}. 5. Aufl. Basel: Beltz Verlag.

Schöne, Helmar (2003): \emph{Die teilnehmende Beobachtung als
Datenerhebungsmethode in der Politikwissenschaft. Methodologische
Reflexion und Werkstattbericht}. In: Forum qualitative Sozialforschung,
Volume 4, No. 2. Nicht paginiert.
\url{http://www.qualitative-research.net/index.php/fqs/article/view/720/1558}
{[}12.06.2014{]}

Swiss Academy for Development (2015): \emph{Empirische Sozialforschung.}
\url{http://www.sad.ch/de/methodik/empirische-sozialforschung} {[}Stand
2.3.2015{]}

Vogel, Bernd \& Woisch, Andreas (2013): \emph{Orte des Selbststudiums.
Eine empirische Studie zur zeitlichen und räumlichen Organisation des
Lernens von Studierenden.} Hannover, HIS Hochschul-Infor\-mations-System
GmbH. \url{http://www.his.de/pdf/pub_fh/fh-201307.pdf} {[}28.04.2014{]}

%autor
\begin{center}\rule{0.5\linewidth}{\linethickness}\end{center}

\textbf{Felix Hüppi} \url{mailto:fehuppi@gmail.com} arbeitet in der
Geschäftsleitung der Pestalozzi Bibliothek Zürich, die Öffentliche
Bibliothek der Stadt. Er hat einen Abschluss als Master of Science in
Business Administration und arbeitet seit 8 Jahren im öffentlichen
Bibliothekswesen.

\end{document}