Vor kurzem wurden drei neue Methoden-Handbücher für
Bibliothekswissenschaftler veröffentlicht (Umlauf u.a., Hg. 2013;
Siegfried u.a. 2013 und Schuldt 2014), die alle auf das Thema
Ethnografie im Bibliotheksbereich eingehen. Mein Beitrag fragt danach,
welches Potential sie jeweils in der Ethnografie sehen, und konfrontiert
zwei Positionen: Erstens, Ethnografie sei primär geeignet, die
Nutzerbeteiligung bei Designprozessen zu unterstützen (Umlauf und
Siegfried). Und zweitens, Ethnografie sei ein möglicher Ansatz für
lokale, sozialwissenschaftliche Forschung in und über Bibliotheken
(Schuldt). Die Autorin vertritt die Position, dass partizipatives Design
einen hervorragenden Ansatz für die Nutzerbeteiligung darstellt, sich
das Potential ethnografischer Methoden für den Bibliotheksbereich darin
jedoch nicht erschöpft. Vielmehr ist Ethnografie auch eine Methode für
die kultur- und sozialwissenschaftliche Forschung in und über
Bibliotheken, weshalb Karsten Schuldts Appell, in Bibliotheken lokal zu
forschen, unbedingt Unterstützung verdient.
