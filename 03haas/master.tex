\documentclass[a4paper,
fontsize=11pt,
%headings=small,
oneside,
numbers=noperiodatend,
parskip=half-,
bibliography=totoc,
final
]{scrartcl}

\usepackage{synttree}
\usepackage{graphicx}
\setkeys{Gin}{width=.4\textwidth} %default pics size

\graphicspath{{./plots/}}
\usepackage[ngerman]{babel}
\usepackage[T1]{fontenc}
%\usepackage{amsmath}
\usepackage[utf8x]{inputenc}
\usepackage [hyphens]{url}
\usepackage{booktabs} 
\usepackage[left=2.4cm,right=2.4cm,top=2.3cm,bottom=2cm,includeheadfoot]{geometry}
\usepackage{eurosym}
\usepackage{multirow}
\usepackage[ngerman]{varioref}
\setcapindent{1em}
\renewcommand{\labelitemi}{--}
\usepackage{paralist}
\usepackage{pdfpages}
\usepackage{lscape}
\usepackage{float}
\usepackage{acronym}
\usepackage{eurosym}
\usepackage[babel]{csquotes}
\usepackage{longtable,lscape}
\usepackage{mathpazo}
\usepackage[flushmargin,ragged]{footmisc} % left align footnote

\usepackage{listings}

\urlstyle{same}  % don't use monospace font for urls

\usepackage[fleqn]{amsmath}

%adjust fontsize for part

\usepackage{sectsty}
\partfont{\large}

%Das BibTeX-Zeichen mit \BibTeX setzen:
\def\symbol#1{\char #1\relax}
\def\bsl{{\tt\symbol{'134}}}
\def\BibTeX{{\rm B\kern-.05em{\sc i\kern-.025em b}\kern-.08em
    T\kern-.1667em\lower.7ex\hbox{E}\kern-.125emX}}

\usepackage{fancyhdr}
\fancyhf{}
\pagestyle{fancyplain}
\fancyhead[R]{\thepage}

%meta
%meta

\fancyhead[L]{C. Haas \\ %author
LIBREAS. Library Ideas, 27 (2015). % journal, issue, volume.
\href{http://nbn-resolving.de/urn:nbn:de:kobv:11-100229826
}{urn:nbn:de:kobv:11-100229826}} % urn
\fancyhead[R]{\thepage} %page number
\fancyfoot[L] {\textit{Creative Commons BY 3.0}} %licence
\fancyfoot[R] {\textit{ISSN: 1860-7950}}

\title{\LARGE{Zwischen Participatory Design und lokaler Forschung – Ethnografie in drei neuen Methodenhandbüchern}} %title %title
\author{Corinna Haas} %author

\setcounter{page}{9}

\usepackage[colorlinks, linkcolor=black,citecolor=black, urlcolor=blue,
breaklinks= true]{hyperref}

\date{}
\begin{document}

\maketitle
\thispagestyle{fancyplain} 

%abstracts
\begin{abstract}
Vor kurzem wurden drei neue Methoden-Handbücher für
Bibliothekswissenschaftler veröffentlicht (Umlauf u.a., Hg. 2013;
Siegfried u.a. 2013 und Schuldt 2014), die alle auf das Thema
Ethnografie im Bibliotheksbereich eingehen. Mein Beitrag fragt danach,
welches Potential sie jeweils in der Ethnografie sehen, und konfrontiert
zwei Positionen: Erstens, Ethnografie sei primär geeignet, die
Nutzerbeteiligung bei Designprozessen zu unterstützen (Umlauf und
Siegfried). Und zweitens, Ethnografie sei ein möglicher Ansatz für
lokale, sozialwissenschaftliche Forschung in und über Bibliotheken
(Schuldt). Die Autorin vertritt die Position, dass partizipatives Design
einen hervorragenden Ansatz für die Nutzerbeteiligung darstellt, sich
das Potential ethnografischer Methoden für den Bibliotheksbereich darin
jedoch nicht erschöpft. Vielmehr ist Ethnografie auch eine Methode für
die kultur- und sozialwissenschaftliche Forschung in und über
Bibliotheken, weshalb Karsten Schuldts Appell, in Bibliotheken lokal zu
forschen, unbedingt Unterstützung verdient.
\end{abstract}

%body
\subsubsection{Einleitung}\label{einleitung}

Seit über zehn Jahren gibt es einen internationalen Trend zur
Ethnografie in Bibliotheken (Haas 2013, 2014; Khoo 2012). Mit der Studie
\enquote{Studying Students: The Undergraduate Research Project}
(Foster/Gibbons 2007) kam er endgültig zum Durchbruch. Diese Studie
entstand an der University of Rochester, New York, unter Leitung der
Kulturanthropologinnen Nancy Foster und Susan Gibbons, und unter
Mitwirkung zahlreicher Bibliotheksmitarbeiter.\footnote{Wegen der
  besseren Lesbarkeit wird im Folgenden die männliche Pluralform
  verwendet.} Sie wurde zum Vorbild für über sechzig ähnliche Projekte
in den USA und international. Inzwischen ist bereits eine
Fortsetzungsstudie erschienen: \enquote{Studying Students: A Second
Look} (Foster 2014).

Kürzlich wurden drei neue Methoden-Handbüchern für
Bibliothekswissenschaftler veröffentlicht, was diese Ausgabe von LIBREAS
motivierte:

\begin{itemize}
\item
  Das \enquote{Handbuch Methoden der Bibliotheks- und
  Informationswissenschaft. Bibliotheks-, Benutzerforschung,
  Informationsanalyse} (K. Umlauf/S. Fühles-Ubach/M. Seadle, Hrsg.,
  2013),
\item
  \enquote{Nutzerorientierte Marktforschung für Bibliotheken. Eine
  Praxiseinführung} (D. Siegfried/S. Nix, 2014) und
\item
  " Bibliotheken erforschen ihren Alltag. Ein Plädoyer" (K. Schuldt
  2014).\footnote{I.F. werden die drei Titel als \enquote{Handbuch},
    \enquote{Praxiseinführung} und \enquote{Plädoyer} referenziert.}
\end{itemize}

Alle drei Handbücher stellen Ethnografie als Forschungsmethodologie vor,
womit sie offenbar in den Kanon der deutschsprachigen LIS-Community
aufgenommen ist.

Doch was wird unter Ethnografie verstanden? Gegenstand meines Beitrages
ist die Frage: Welches Potential sehen die Methoden-Publikationen in der
Ethnografie? Für welche Untersuchungsfelder empfehlen sie ethnografische
Methoden, und zur Lösung welcher Aufgaben sollen diese beitragen?

Meines Erachtens kommen unterschiedliche Auffassungen von Ethnografie in
den jeweiligen Publikationen zum Ausdruck, die ich einander gegenüber
stellen werde. Einerseits wird die Auffassung vertreten, Ethnografie sei
ein Instrument der Benutzerforschung und könne, nach dem Vorbild der
Studie \enquote{Studying Students} (Foster/Gibbons 2007, s.o.), in
Designprozessen eingesetzt werden (\enquote{Handbuch} und
\enquote{Praxiseinführung}). Andererseits begegnen wir der Auffassung,
Ethnografie sei ein methodischer Ansatz für die lokale
sozialwissenschaftliche Forschung in Bibliotheken (\enquote{Plädoyer}).

Doch bevor ich auf \enquote{Handbuch}, \enquote{Praxiseinführung} und
\enquote{Plädoyer} eingehe, sei zunächst eine grundsätzliche Frage
behandelt: Warum ist Ethnografie relevant für Bibliotheken? Dies erklärt
Ursula Schulz, deren Name für eine lange Tradition der
Anwenderpartizipation im Katalog- und Webdesign steht,\footnote{Ursula
  Schulz war von 1992 bis 2013 Professorin an der HAW Hamburg, Fakultät
  DMI, Department Information. Auf ihrer Webseite sind zahlreiche
  Projekte zur Anwenderpartizipation nachgewiesen:
  http://www.bui.haw-hamburg.de/pers/ursula.schulz/} im Folgenden sehr
anschaulich:

\begin{quote}
Bibliotheken haben eine lange Geschichte frustrierter Erwartungen an
ihre Kunden. Bibliotheken arbeiten nach eigenen Standards, haben ein
eigenes Verständnis von Qualität und Dienstleistungen, die Bibliotheken
bieten sollten. Sie haben ein eigenes professionelles Selbstverständnis
entwickelt, eine eigene Berufskultur. Ihre Kunden werden ‚Benutzer`
genannt, Benutzer, die lernen sollen, sich in dieser ihnen fremden
Kultur zu bewegen. {[}\ldots{}{]} Die Geschichte frustrierter
Erwartungen an die Kunden kam allerdings vor wenigen Jahren an einen
interessanten Wendepunkt: An den Komfort des nahtlosen Discovery to
Delivery Prozesses vom heimischen Schreibtisch aus gewöhnt, glauben
viele Kunden ohne Bibliothek auskommen zu können. Internetdienste wie
Google und Wikipedia entsprechen der Informationskultur der meisten
Nutzer und prägen sie nachhaltig. Bibliotheken begannen vor rund zehn
Jahren, sich ethnographischer Methoden zu bedienen (Khoo 2012), um
ihrerseits von der Arbeitskultur ihrer Klientel zu lernen und ihre
Dienstleistungen darauf auszurichten. Der Druck zur Anpassung an die
Kultur der Anderen verläuft zunehmend in die entgegengesetzte Richtung.
Bibliothekare müssen sich fragen: Was können Bibliotheken heute bedeuten
und wie sehen Dienstleistungen aus, die wirklich gebraucht werden?
(Schulz 2013, 2)
\end{quote}

Folglich unterstützen ethnografische Methoden den Perspektivwechsel von
der Experten- zur Nutzersicht. Ethnologie und Kulturanthropologie haben
Methoden entwickelt, um die Kultur einer Gruppe zu untersuchen und aus
der Innenperspektive zu verstehen.\footnote{Aus einer Definition von
  Ethnografie: \enquote{Ethnography entails the close and prolonged
  observation of a particular social group. The ethnographer is not
  concerned to describe the behavior of the members of the group, but
  rather to understand the \textbf{culture} of that group from within.}
  (Edgar/Sedgwick 2008, 115)} Sich diese zu eigen zu machen, kann
Bibliotheken dabei helfen, wirklich kundenorientiert zu arbeiten.

\subsubsection{Participatory Design in \enquote{Studying
Students}}\label{participatory-design-in-studying-students}

Zwei gegenläufige Entwicklungen stellen Bibliotheken heute vor besondere
Herausforderungen: Während einerseits das wachsende Angebot an
Online-Ressourcen zunehmend im Fernzugriff genutzt wird, füllen sich
andererseits die Lesesäle und Publikumsbereiche. Daraus ergeben sich
unter anderem die Aufgaben, die Online-Ressourcen und -Plattformen sowie
die Raumangebote der Bibliotheken ständig weiter zu entwickeln. Für
diese Aufgaben führt \enquote{Studying Students} (Foster/Gibbons 2007)
einige praktikable Ansätze vor, was die große Resonanz auf diese Studie
zum Teil erklärten dürfte.

Foster und Gibbons haben hierin zahlreiche Aspekte studentischer
Arbeitskultur untersucht. Sie unterscheidet sich erheblich von der
Arbeitskultur der Bibliotheksmitarbeiter! So betreten zum Beispiel die
meisten Studierenden die Universitätsbibliothek rund acht Stunden nach
den Auskunftsbibliothekaren, nach all ihren Lehr- und
Sportveranstaltungen auf dem Campus, um an ihren Research Papers zu
arbeiten. Dann gehen die Auskunftsbibliothekare schon wieder nach Hause
-- ein absurder \enquote{Schichtwechsel}, dem dann mit erweiterten
Abendöffnungszeiten (\enquote{Night Owl Librarians}) abgeholfen wurde
(Foster/Gibbons 2007, Kap.3, 16-19). Die studentische Arbeitskultur
unterscheidet sich zudem in vielen Punkten von den Vorannahmen der
Bibliotheksmitarbeiter. Auch dies zeichnete sich in ersten,
explorierenden Interviews mit den Studierenden ab und führte dazu, sie
in bestimmte Planungen mit einzubeziehen. Um die Formen der Beteiligung
von Studierenden in \enquote{Studying Students} etwas anschaulicher zu
machen, beschreibe ich kurz den Ablauf eines Design Workshops:

Als die Neugestaltung von Arbeitsbereichen in einer von Foster und
Gibbons betrachteten Bibliothek anstand, lud das Renovierungsteam der
Bibliothek eine Gruppe von rund zwanzig Studierenden zu einem
\enquote{Design Workshop} ein (s. Foster/Gibbons 2007, Kap. 4, 20.29).
Die Teilnehmer sollten innerhalb von zwei Stunden mit Stiften, Markern
und Klebezetteln auf großen Papierbögen ihre Traumbibliothek entwerfen,
die sie ganz nach Belieben mit Mobiliar, Raumteilern, Technik und
Personal ausstatten konnten. Anschließend wurden die Entwürfe mit den
Studierenden durchgesprochen und dann Gemeinsamkeiten herausgefiltert:
Einrichtungselemente wie offene Kamine, Sitzsäcke, Sofas und
Massageliegen wurden zum Beispiel als Hinweise auf das Bedürfnis nach
einer \enquote{comfy area}, also einem wohnzimmerartigen Bereich,
kategorisiert und eine \enquote{comfy area} dann in die weitere
Einrichtungsplanung übernommen. Ähnlich wie in diesem Beispiel wurden
weitere Anforderungen an den Bibliotheksraum (Ruhe, Anregung,
Möglichkeiten zur Gruppenarbeit\ldots{}) identifiziert und entsprechende
Zonen geschaffen. Häufig wichen dabei die Wünsche der Studierenden von
den Vorannahmen der Bibliothekare und Architekten ab. Indem die
Anregungen der Studierenden aufgenommen wurden, konnten also
Planungsfehler vermieden und die Benutzerorientierung verbessert werden.

Solche Design Workshops fanden auch zur Überarbeitung der Website statt,
oder, in der Nachfolgestudie (Foster 2014), für die Entwicklung von Apps
für mobile Endgeräte.

Als weitere Formen studentischer Beteiligung kamen \enquote{Photo
Surveys} (Kap.6, 40-47) und \enquote{Mapping Diaries} (Kap. 7, 48-54)
zum Einsatz. Photo Surveys sind visuell gestützte Leitfadeninterviews.
Die Probanden wurden dabei gebeten, anhand einer Themenliste
(\enquote{Arbeitsmittel, die Du immer dabei hast}; \enquote{ein Ort in
der Bibliothek, an dem Du Dich verloren fühlst} und so weiter) Fotos zu
machen, die in einem anschließenden Face-to-Face-Interview gemeinsam
angesehen und besprochen wurden. Dabei kamen viele Aspekte zur Sprache,
nach denen die Interviewerin von sich aus nicht gefragt hätte.

In \enquote{Mapping Diaries} zeichneten die Probanden ihre täglichen
Wege auf dem Campus ein -- zwischen Wohnheim, Hörsälen, Computer Lab,
Bibliothek und anderen Orten (eine Methode, die wohl nur für
Campusuniversitäten in Frage kommt). Daraus konnten Erkenntnisse über
ihren Tagesablauf (kein Tag wie der andere, immer \enquote{on the go} et
cetera) gewonnen und Schlüsse für Serviceverbesserungen gezogen werden,
zum Beispiel wo abschließbare Schränke für Laptops aufgestellt werden
sollten.

Methoden, bei denen Studierende selbst etwas herstellten -- Fotos,
Pläne, Karten und so weiter -- werden auch unter dem Begriff
\enquote{Cultural Probes}, kulturelle Proben subsummiert.

Wie deutlich werden sollte: \enquote{Participatory Design}, also die
Beteiligung von Nutzern an Gestaltungsprozessen, ist ein zentrales Thema
der vorgestellten Studie. Unter dem Aspekt des Designs von physischen
und virtuellen Bibliotheksräumen wird \enquote{Studying Students} auch
im \enquote{Handbuch} und in der \enquote{Praxiseinführung} vorgestellt.

\paragraph{Ethnomethodologie im
\enquote{Handbuch}}\label{ethnomethodologie-im-handbuch}

Das \enquote{Handbuch Methoden der Bibliotheks- und
Informationswissenschaft} widmet dem Thema Ethnomethodologie\footnote{Ethnomethodologie
  ist mit Ethnografie eigentlich nicht identisch; der Ansatz stammt aus
  der Soziologie und ist an die Kulturanthropologie angelehnt
  (Edgar/Sedgwick 2008, 116ff.). So scharfe Unterscheidungen scheinen
  mir jedoch in Bezug auf LIS-Anwendungen nicht hilfreich, weshalb ich
  sie hier übergehe und Seadles Begriff \enquote{Ethnomethodologie}
  übernehme.} in einer Reihe von Kapiteln, die auf komplexe
Fragestellungen ausgerichtet sind, einen Beitrag von über zwanzig Seiten
(315-337). Sein Verfasser Michael Seadle, der sich seit langem für
Ethnologie im LIS-Bereich einsetzt (Seadle 1998, 2000, 2007, 2011), gibt
darin eine methodische Einführung für Bibliothekare, die eigene
Forschungen anstellen wollen. Dabei stützt er sich auf die ethnologische
Theorie der Dichten Beschreibung von Clifford Geertz (1973) und die
Forschungen von Nancy Foster und Susan Gibbons. Diese gelten ihm also
nicht nur als Anwendungsbeispiele, sondern als Maßstab für
Ethnomethodologie im LIS-Bereich. Techniken der Datenerhebung --
verschiedene Formen der Befragung, des Interviews und der Beobachtung --
werden ausführlich vorgestellt und anhand der Rochester-Studie von 2007
und zweier weiterer Projekte in Rutgers (2009) und Connecticut (2012)
illustriert. Aus \enquote{Studying Students} werden die drei oben
beschriebenen Methoden vorgestellt, also Design Workshop, Foto-Essays
und Mapping Diaries.

Ein eigener Absatz ist dem Thema \enquote{Bibliothekare und Ethnologie}
gewidmet (331f.). Hier betont Seadle die besondere Eignung von
Bibliothekaren als Ethnografen, weil viele einen
geisteswissenschaftlichen Background hätten und in ihrem Berufsalltag
Beobachtungen anstellen könnten, aus denen sich dann Forschungsfragen
entwickeln ließen.

Für unsere Frage -- Welche Rolle spielt Ethnografie in Bibliotheken? --
ist hervorzuheben, dass Seadle deutlich zwischen der
\enquote{LIS-Anwendung ethnologischer Methoden} und \enquote{Ethnologie
selbst} unterscheidet (327). Das primäre Anwendungsfeld für Ethnologie
im LIS-Bereich sieht er in der Marktforschung. So leitet er
LIS-Anwendungen von Ethnologie von der Organisationsethnologie ab, die
auch Managementprobleme untersuche, und hebt die Stärke ethnologischer
Methoden im Marketingbereich besonders hervor (315). Er betont auch,
dass Ethnografie im Bibliothekskontext grundsätzlich anwendungsbezogen
sei: \enquote{In LIS hat man fast immer einen konkreten Fall, für den
eine Lösung gefunden werden soll} (317). Diese dezidierte
Anwendungsorientierung weicht allerdings ab von dem Anspruch des
\enquote{Handbuchs}, wie Konrad Umlauf ihn in der Einleitung formuliert:

\begin{quote}
\enquote{Beim Stichwort Methoden denken Praktiker {[}\ldots{}{]} an
Methoden, die in der Praxis Probleme lösen. Um solche Methoden geht es
in diesem Handbuch nicht. Vielmehr behandeln die Beiträge {[}\ldots{}{]}
Forschungsmethoden die zur Anwendung kommen, um neue Erkenntnisse zu
gewinnen.} (21)
\end{quote}

Auch wenn Umlauf dann ergänzt, dass die \enquote{Forschungsergebnisse
mindestens teilweise letztlich auch wieder im Anwendungsfeld
{[}\ldots{}{]} ausgewertet werden können} (ebd.), erscheint ein so enger
Anwendungsbezug, wie er in Seadles Beitrag zum Ausdruck kommt, nicht als
zwingend.

Zusammenfassend kann jedenfalls festgestellt werden: Seadle sieht in
seinem Beitrag für das \enquote{Handbuch} das Potential der Ethnografie
im Marketing, ihre Untersuchungsfelder im Participatory Design und ihre
Aufgaben in der Lösung von Managementfragen.

\paragraph{Nutzerorientierte Gestaltung in der
\enquote{Praxiseinführung}}\label{nutzerorientierte-gestaltung-in-der-praxiseinfuxfchrung}

Eine ähnliche Auffassung vertreten Doreen Siegfried und Sebastian Nix in
der \enquote{Praxiseinführung}. Diese richtet sich an Bibliothekare, die
selbst Nutzerstudien durchführen wollen. Die Begriffe
\enquote{Ethnologie} und \enquote{Ethnografie} sucht man im
Inhaltsverzeichnis zunächst vergeblich, entdeckt sie dann aber doch --
erwartungsgemäß -- in einem Kapitel über die \enquote{Nutzerorientierte
Gestaltung von physischen und virtuellen Bibliotheksräumen}
(Siegfried/Nix 2013,127-146). Aus dem Titel ergibt sich bereits, dass
Ethnografie auch hier als Instrument der Kunden- und Marktforschung im
Anwendungsfeld Participatory Design vorgestellt wird. Den Bezug zur
Ethnologie beschreiben Siegfried und Nix wie folgt:

\begin{quote}
\enquote{Die Methoden, die hier zum Einsatz kommen, sind vielfältig und
eher qualitativ orientiert. Der Grund dafür liegt darin, dass man oft
einen tiefen Einblick in die Wünsche und Bedürfnisse der Nutzer gewinnen
und diese {[}\ldots{}{]} aktiv an der Gestaltung bibliothekarischer
Angebote beteiligen möchte. Deshalb basieren entsprechende Ansätze
{[}\ldots{}{]} auch auf ethnografischen Methoden. Sie sind also
Disziplinen wie der Ethnologie und Kulturanthropologie entlehnt denen es
darum geht, Menschen in ihrem spezifischen sozialen und kulturellen
Umfeld zu verstehen. Bei der bibliothekarischen Nutzerforschung will man
{[}\ldots{}{]} unterschiedliche Gruppen von Nutzern {[}\ldots{}{]} mit
ihren jeweiligen Gewohnheiten und Bedürfnissen besser verstehen, um
ihnen {[}\ldots{}{]} bedarfsgerechte Angebote machen zu können. Im Kern
ethnografischer Methoden stehen beobachtungsbasierte Verfahren, die
nicht selten mit Befragungen kombiniert werden. Solche Ansätze
{[}\ldots{}{]} eignen sich für die die nutzerorientierte Gestaltung
physischer und virtueller Bibliotheksräume {[}\ldots{}{]}.} (17)
\end{quote}

Auch die Autoren der \enquote{Praxiseinführung} heben also den
Vorbildcharakter von \enquote{Studying Students} hervor. Sie stellen
zwei Methoden daraus vor, \enquote{Design Workshop} und
\enquote{Kulturelle Proben}, womit hier konkret Photo Surveys
angesprochen sind. Anders als Seadle, der Beispiele aus den USA
vorstellt, haben sie Anwendungsbeispiele aus deutschen Bibliotheken in
Oldenburg (Schoof 2010)\footnote{Kerstin Schoof hat Participatory Design
  und Photo Surveys aus der Rochester-Studie aufgenommen und an der
  Universitätsbibliothek Oldenburg angewendet.} und Hamburg gewählt, was
den deutschsprachigen Lesern entgegen kommt. Viele Illustrationen, wie
Abbildungen von Einrichtungsplänen aus Design Workshops, runden die
Beschreibung anschaulich ab.

Zusammenfassend stellen wir fest: Auch die \enquote{Praxiseinführung}
sieht das Potential der Ethnografie auf dem Feld einer
kundenorientierten Marktforschung, ihr Anwendungsfeld im Participatory
Design und ihre Aufgabe in der Gestaltung physischer und virtueller
Bibliotheksräume. Gegenüber Seadle betonen sie den Eigenwert der
Ethnologie etwas weniger und die Beteiligung der Nutzer etwas mehr; doch
insgesamt liegen sie auf der selben Linie.

\subsubsection{Ethnografie im
\enquote{Plädoyer}}\label{ethnografie-im-pluxe4doyer}

Eine ganz andere Linie verfolgt Karsten Schuldt mit seinem Buch
\enquote{Bibliotheken erforschen ihren Alltag. Ein Plädoyer} (2014). Wir
verlassen hiermit den Bereich Nutzerbeteiligung / Bibliotheksdesign und
wenden uns einer als Kultur- und Sozialwissenschaft verstandenen
Bibliothekswissenschaft zu.

Mit dem \enquote{Plädoyer} verfolgt Schuldt das Ziel, besonders
Bibliothekare an Öffentlichen Bibliotheken zu eigener Forschung
aufzurufen, ja, aufzurütteln. Ausgangspunkt dieses Appells ist die
Feststellung, dass öffentliche Bibliotheken besonders wissenschaftsfern
seien (Schuldt 2014, 7): So würden Publikationen über Öffentliche
Bibliotheken von Projektbeschreibungen dominiert, die Frage nach der
theoretischen Basis aber selten berührt (ebd.). Dabei wollten viele
Bibliothekare nicht ausschließlich Praxisbeispiele sammeln, sondern
seien -- jedenfalls noch während ihrer Ausbildung -- durchaus an
wissenschaftlichen, über ihr Tagesgeschäft hinausreichenden
Fragestellungen interessiert. Doch einmal in der Praxis angelangt,
forschten sie nicht mehr (10) -- mit der Konsequenz, dass die Verbindung
zwischen Wissenschaft und Praxis verloren gehe und die Wissenschaft sehr
unzureichende Einblicke in die Realität Öffentlicher Bibliotheken hätte.

Doch warum sollte überhaupt in Öffentlichen Bibliotheken geforscht
werden? Weil es \enquote{gesellschaftlich relevante Einrichtungen} (13)
seien, so Schuldts Kernargument, das er wie folgt auffächert:
Bibliotheken würden von der Gesellschaft getragen, übten selbst eine
gesellschaftliche Wirkung aus, und man könne dort beobachten, wie
gesellschaftliche Prozesse auf lokaler Ebene wirksam werden. Konkret
spielten Öffentliche Bibliotheken zum Beispiel eine Rolle als Teil von
Lernnetzwerken, als Erlebnisort für Kinder oder als Orte
zivilgesellschaftlichen Engagements (13ff.).

Schuldt unterbreitet einige Vorschläge, worüber in Öffentlichen
Bibliotheken geforscht werden könnte: etwa über die Rolle von
Bibliotheken im Alltag und in der Biografie von Menschen, oder darüber,
wie unterschiedlich Menschen mit verschiedenen sozialen Hintergründen in
der Bibliothek agierten (15ff.). Solche Themen kennt man aus der
Europäischen Ethnologie und Empirischen Kulturwissenschaft (auch auf
Bibliotheken bezogen), allerdings nicht aus der bibliothekarischen
Ausbildung oder (gar) der Praxis in Öffentlichen Bibliotheken. In
anderen Ländern wie zum Beispiel Frankreich, werden jedoch
ethnografische Studien zu solchen Themen veröffentlicht -- an
Wissenschaftlichen (z. B. Roselli/Perrenoud 2010) und Öffentlichen
Bibliotheken (z. B. Paugham/Giorgetti 2013).

Was unterscheidet aber die lokale Forschung in Bibliotheken, Schuldt
zufolge, von der Benutzerforschung, die im \enquote{Handbuch} und der
\enquote{Praxiseinführung} vertreten werden? Vor allem ein
Erkenntnisinteresse (also nicht eine Projektaufgabe) und eine
Ergebnisoffenheit (also nicht zwingend ein unmittelbarer
Anwendungsbezug). Weder das Erkenntnisinteresse noch die
Ergebnisoffenheit seien in Evaluationen und Projektbeschreibungen
gegeben. Dennoch erzeuge \enquote{echte} Forschung mit der Zeit
ebenfalls ein „Funktionswissen``, dass man für die bibliothekarische
Arbeit nutzen könne (s. Kap. 3, Grundlagen der Forschung, bes. 23-36).

Als Vorbilder für eine lokale Forschung in Bibliotheken zieht Schuldt
die Aktionsforschung in Schulen und der Sozialen Arbeit sowie der
Evidence Based Librarianship in Kanada heran (119-130). Obwohl die
Ethnologie nicht eingehend besprochen wird, macht der Autor doch
deutlich, dass er ethnografische Methoden als für die lokale Forschung
prädestiniert sieht:

\begin{quote}
\enquote{Uns interessiert die theoretische Möglichkeit, Forschungen aus
unterschiedlichen Bildungseinrichtungen und -gängen miteinander zu
verbinden. Die Europäische Ethnologie\footnote{Schuldt bezieht sich hier
  auf die Europäische Ethnologie, da sie, anders als die Ethnologie
  nicht die fremde, sondern die eigene Kultur zum Forschungsgegenstand
  macht (Kaschuba 2003).} interessiert uns vor allem, weil sie Methoden
entwickelt hat, das Verhalten der Individuen alleine und in Gruppen zu
verstehen.} (58)
\end{quote}

An anderer Stelle wird die Rolle von Beobachtungsverfahren in der
Ethnologie angesprochen:

\begin{quote}
\enquote{{[}I{]}n der Ethnologie bedeutet der Einsatz von Beobachtungen
immer wieder eine {[}\ldots{}{]} Wiederholung der Schritte Planung --
Beobachtung -- Reflektion. Ziel ist es durch Beobachtungen etwas über
die Realität zu erfahren, was wir mit anderen Methoden nicht erfahren
können.} (95)
\end{quote}

Sicher hätte der Autor das Potential der Ethnografie für die lokale
Forschung eingehender beschreiben können. Vielleicht hätte sich mit
Clifford Geertz argumentieren lassen, dass sich die Ethnologie
\enquote{von der sehr intensiven Bekanntschaft mit äußerst kleinen
Sachen her}, die sie mikroskopisch beschreibt, ganzen Gesellschaften
nähert (Geertz 1997, 30). Übertragen auf unser Feld hieße das, dass die
detaillierte Untersuchung der Interaktionen in einer Bibliothek über
sich selbst hinausweist, indem sie auch ihren gesellschaftlichen Kontext
erhellt.

Es ist jedoch für Schuldts \enquote{Plädoyer} charakteristisch, dass es
auf Disziplinen und Methoden eher hinweist als sie eingehend
vorzustellen. Denn das primäre Ziel der Veröffentlichung ist, Forschung
in Öffentlichen Bibliotheken erst einmal generell zu postulieren.
Entsprechend fasst Schuldt im letzten Satz des \enquote{Plädoyers}
zusammen:

\enquote{Das war das Ziel des Buches: Einfach mal laut zu sagen, dass
Aktionsforschung, Evidence-based Librarianship, Handlungsforschung,
Forschung in der Sozialen Arbeit und ähnliche Ansätze als Vorbild für
die Forschung in Öffentlichen Bibliotheken dienen können. \emph{Wir
sollten es versuchen.}} (S. 150)

Im Unterschied zu \enquote{Handbuch} und \enquote{Praxiseinführung}
sieht das Plädoyer also das Potential der Ethnografie -- auch im
Bibliotheksbereich -- in einer sozialwissenschaftlichen, lokalen
Forschung.

Es ist zu hoffen, dass das \enquote{Plädoyer} eine Diskussion über die
Rolle des Bibliothekars als Forscher -- und Ethnograf -- anstößt, und
dass es Bibliothekarinnen und Bibliothekare zu eigener Forschung
ermutigt. Gerade Bibliothekare an Öffentlichen Bibliotheken sind
außerordentlich aktiv und einfallsreich, wenn es um lokale Aktivitäten
geht -- und um die Kooperationen und Fördermöglichkeiten, die Forschung
in diesem Rahmen überhaupt ermöglichen.

\subsubsection{Schlussbetrachtung}\label{schlussbetrachtung}

Drei neue Handbücher tragen also dazu bei, dass methodische Fragen
künftig besser in die bibliotheks- und informationswissenschaftliche
Aus- und Fortbildung integriert werden können. Gemeinsam ist ihnen das
Verdienst, Forschungsansätze und -methoden einem deutschsprachigen
Publikum näher zu bringen, das bisher nur auf sehr wenig Fachliteratur
zurückgreifen konnte. Die Publikationen stellen zahlreiche quantitative
und qualitative Methoden vor, sowie Methodenkombinationen, wie sie in
der Ethnologie üblich sind. Ethnografie in Bibliotheken ist ein
Trendthema, das in den drei Publikationen jedoch unter verschiedenen
Gesichtspunkten betrachtet wird. Während das \enquote{Handbuch} und die
\enquote{Praxiseinführung} ethnografische Ansätze Designprozessen
zuordnen, sieht das \enquote{Plädoyer} die Rolle des ethnografischen
Instrumentariums bei einer sozialwissenschaftlichen
Bibliotheksforschung, die auch aus den Bibliotheken heraus stattfinden
sollte.

In der Anwendung von Methoden, die der Ethnologie entlehnt sind, wie das
Participatory Design, zeigt sich, was Nutzerforschung im besten Falle
sein kann: Ein Hebel zur Veränderung. In der Einbeziehung von Kunden in
Designprozesse kommt eine wertschätzende Haltung diesen gegenüber zum
Ausdruck, vielleicht sogar ein Wandel im Verhältnis zwischen
\enquote{Experten} und \enquote{Laien}, der sich auch in anderen
gesellschaftlichen Feldern ankündigt. Ansonsten müssen \enquote{Create
your own workspace}- Aktivitäten vielleicht gar nicht unbedingt aus der
Ethnologie hergeleitet werden (was oft auch nicht geschieht) -- es ist
zumindest zur fragen, ob Verfahren der User Participation in
bibliothekarischen und außerbibliothekarischen Feldern wie
Produktdesign, Offene Innovation oder Usability- Forschung und so weiter
wirklich so eindeutig auf die Ethnologie zurückzuführen sind.

Lokale Forschung könnte hingegen das Potential der Ethnografie für den
Bibliotheksbereich wirklich zum Tragen bringen. Das „Plädoyer" klagt
einen Anspruch ein, den auch die Herausgeber des \enquote{Handbuchs} an
die Disziplin stellen, dann aber gleich wieder relativieren (siehe oben)
und der in der Bibliothekswissenschaft noch entschlossener verfolgt
werden könnte. Ich zitiere abschließend noch einmal aus der Einleitung
von Konrad Umlauf: \enquote{Beim Stichwort Methoden denken Praktiker
{[}\ldots{}{]} an Methoden, die in der Praxis Probleme lösen. Um solche
Methoden geht es in diesem Handbuch nicht. Vielmehr behandeln die
Beiträge {[}\ldots{}{]} Forschungsmethoden die zur Anwendung kommen, um
neue Erkenntnisse zu gewinnen.} (21) Lokale Forschung in Bibliotheken
könnte Erkenntnisse hervorbringen, die Theorie und Praxis verbinden und
für beide von Nutzen sind.

\subsubsection{Literaturverzeichnis}\label{literaturverzeichnis}

EDGAR, ANDREW / SEDGWICK, PETER (2008): Cultural Theory: The Key
Concepts. London: Routledge

FOSTER, NANCY / GIBBONS, SUSAN (Eds.) (2007): Studying Students. The
Undergraduate Research Project at the University of Rochester. Chicago:
American Library Association. Online verfügbar unter
\href{http://hdl.handle.net/1802/7520}{\emph{http://hdl.handle.net/1802/7520}}

FOSTER, NANCY (Ed.) (2013) Studying Students - a Second Look. Chicago:
American Library Association. Online verfügbar unter
\href{http://hdl.handle.net/1802/28781}{\emph{http://hdl.handle.net/1802/28781}}

GEERTZ, CLIFFORD (1973): Thick Description. Toward an Interpretive
Theory of Culture. In: Ders. (1973), The Interpretation of Cultures, New
York: Basic Books, S. 3-30

GEERTZ, CLIFFORD (1997): Dichte Beschreibung. Beiträge zum Verstehen
kultureller Systeme. Frankfurt a.M.: Suhrkamp {[}Deutsche Übersetzung
von Geertz (1973){]}

HAAS, CORINNA (2013): \enquote{Spielen die jetzt Soziologen?}
Nutzerstudien mit ethnografischen Methoden. In: 027.7 Zeitschrift für
Bibliothekskultur 1,3 (2013): Vom Willen zu verstehen, S. 101-105

HAAS, CORINNA (2014): Wozu Ethnografie in Bibliotheken? In: Bibliothek
Forschung und Praxis 38,2. S. 185-189. Online verfügbar unter
\href{http://dx.doi.org/10.1515/bfp-2014-0023}{\emph{http://dx.doi.org/10.1515/bfp-2014-0023}}

KASCHUBA, WORLFGANG (2012): Einführung in die Europäische Ethnologie.
4., aktual. Ausgabe. München: Beck

KHOO, MICHAEL / ROZAKLIS, LIZ / HALL, CATHERINE (2012): A Survey of the
Use of Ethnographic Methods in the Study of Libraries and Library Users.
In: Library and Information Research 34,2. S. 82-91. Online verfügbar
unter
\href{http://dx.doi.org/10.1016/j.lisr.2011.07.010}{\emph{http://dx.doi.org/10.1016/j.lisr.2011.07.010}}

PAUGAM, SERGE / GIORGETTI, CAMILA (2013): Des pauvres à la bibliothèque.
Enquête au Centre Pompidou. Paris

ROSELLI, MARIANGELA / PERRENOUD, PIERRE (2010): Du lecteur à l'usager.
Ethnographie d'une Bibliothèque Universitaire. Toulouse: Presses
Universitaires de Mirail

SCHOOF, KERSTIN (2010): Kooperatives Lernen als Herausforderung für
Universitätsbibliotheken. Veränderungen in der Konzeption und Nutzung
von Lernräumen. Online verfügbar unter
\href{http://nbn-resolving.de/urn:nbn:de:kobv:11-100113210}{\emph{http://nbn-resolving.de/urn:nbn:de:kobv:11-100113210}}

SCHULDT, KARSTEN (2014): Bibliotheken erforschen ihren Alltag. Ein
Plädoyer. -- Berlin: Simon-Verlag für Bibliothekswissen

SCHULZ, URSULA (Hrsg.) (2013): Service nach Maß -- Eine Bibliothek für
die Informationskultur der Studierenden am Department Design. Online
verfügbar unter
\href{http://www.bui.haw-hamburg.de/pers/ursula.schulz/publikationen/ethnographie_infokult.pdf}{\emph{http://www.bui.haw-hamburg.de/pers/ursula.schulz/publikationen/ethnographie\_infokult.pdf}}

SEADLE, MICHAEL (1998): The Raw and the Cooked among Librarians. In:
Library Hi Tech 16,3-4. S. 7-11. Online verfügbar unter
\href{http://dx.doi.org/10.1108/07378839810306025}{\emph{http://dx.doi.org/10.1108/07378839810306025}}

SEADLE, MICHAEL (2000): Project Ethnography: An Anthropological Approach
to Assessing Digital Library Services. In: Library Trends 49,2. S.
370-385. Online verfügbar unter
\href{http://hdl.handle.net/2142/8338}{\emph{http://hdl.handle.net/2142/8338}}

SEADLE, MICHAEL (2007): Anthropologists in the Library: A Review of
Studying Students. In: Library Hi Tech 25,4. S. 612-619. Online
verfügbar unter
\href{http://dx.doi.org/10.1108/07378830710840545}{\emph{http://dx.doi.org/10.1108/07378830710840545}}

SEADLE, MICHAEL (2011): Research Rules for Library Ethnography. In:
Library Hi Tech 29,3. S. 409-411. Online verfügbar unter
\href{http://dx.doi.org/10.1108/07378831111174378}{\emph{http://dx.doi.org/10.1108/07378831111174378}}

SEADLE, MICHAEL (2012): Ethnomethodologie, in: Umlauf, K. u.a. (2013),
S. 315-337

SIEGFRIED, DOREEN / NIX, SEBASTIAN JOHANNES (2013): Nutzerbezogene
Marktforschung für Bibliotheken: Eine Praxiseinführung. Berlin: De
Gruyter

UMLAUF, KONRAD / FÜHLES-UBACH, SIMONE / SEADLE, MICHAEL (Hrsg.) (2013):
Handbuch Methoden der Bibliotheks- und Informationswissenschaft.
Bibliotheks-, Benutzerforschung, Informationswissenschaft. Berlin: De
Gruyter

%autor
\begin{center}\rule{0.5\linewidth}{\linethickness}\end{center}

\textbf{Corinna Haas} ist Wissenschaftliche Bibliothekarin am ICI
Institute for Cultural Inquiry Berlin, Kontakt:
\url{corinna.haas@ici-berlin.org}

\end{document}