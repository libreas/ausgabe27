\documentclass[a4paper,
fontsize=11pt,
%headings=small,
oneside,
numbers=noperiodatend,
parskip=half-,
bibliography=totoc,
final
]{scrartcl}

\usepackage{synttree}
\usepackage{graphicx}
\setkeys{Gin}{width=.4\textwidth} %default pics size

\graphicspath{{./plots/}}
\usepackage[ngerman]{babel}
\usepackage[T1]{fontenc}
%\usepackage{amsmath}
\usepackage[utf8x]{inputenc}
\usepackage [hyphens]{url}
\usepackage{booktabs} 
\usepackage[left=2.4cm,right=2.4cm,top=2.3cm,bottom=2cm,includeheadfoot]{geometry}
\usepackage{eurosym}
\usepackage{multirow}
\usepackage[ngerman]{varioref}
\setcapindent{1em}
\renewcommand{\labelitemi}{--}
\usepackage{paralist}
\usepackage{pdfpages}
\usepackage{lscape}
\usepackage{float}
\usepackage{acronym}
\usepackage{eurosym}
\usepackage[babel]{csquotes}
\usepackage{longtable,lscape}
\usepackage{mathpazo}
\usepackage[flushmargin,ragged]{footmisc} % left align footnote

\usepackage{listings}

\urlstyle{same}  % don't use monospace font for urls

\usepackage[fleqn]{amsmath}

%adjust fontsize for part

\usepackage{sectsty}
\partfont{\large}

%Das BibTeX-Zeichen mit \BibTeX setzen:
\def\symbol#1{\char #1\relax}
\def\bsl{{\tt\symbol{'134}}}
\def\BibTeX{{\rm B\kern-.05em{\sc i\kern-.025em b}\kern-.08em
    T\kern-.1667em\lower.7ex\hbox{E}\kern-.125emX}}

\usepackage{fancyhdr}
\fancyhf{}
\pagestyle{fancyplain}
\fancyhead[R]{\thepage}

%meta
%meta

\fancyhead[L]{K. Schuldt, R. Mumenthaler \\ %author
LIBREAS. Library Ideas, 27 (2015). % journal, issue, volume.
\href{http://nbn-resolving.de/urn:nbn:de:kobv:11-100229836
}{urn:nbn:de:kobv:11-100229836}} % urn
\fancyhead[R]{\thepage} %page number
\fancyfoot[L] {\textit{Creative Commons BY 3.0}} %licence
\fancyfoot[R] {\textit{ISSN: 1860-7950}}

\title{\LARGE{Forschungsmethoden in die Praxisausbildung einbinden? Ansätze an der HTW Chur}} %title %title
\author{Karsten Schuldt \& Rudolf Mumenthaler} %author

\setcounter{page}{19}

\usepackage[colorlinks, linkcolor=black,citecolor=black, urlcolor=blue,
breaklinks= true]{hyperref}

\date{}
\begin{document}

\maketitle
\thispagestyle{fancyplain} 

%abstracts

%body
\section*{Einleitung: Die Situation an der HTW
Chur}\label{einleitung-die-situation-an-der-htw-chur}

Die Hochschule für Technik und Wirtschaft (HTW) Chur in der
deutschsprachigen Schweiz bietet unter anderem einen Studiengang
(Bachelor, Master, Master of Advanced Studies(MAS)) in
Informationswissenschaft an. Im Rahmen dieses Studienganges wird -
aktuell als Vertiefungsrichtung, in Zukunft als Major -
Bibliothekswissenschaft unterrichtet, wobei die beiden Autoren dieses
Textes zusammen mit einer weiteren Kollegin den Grossteil des
Unterrichts in diesem Fach übernehmen.

Es existieren in der Schweiz wenige andere Möglichkeiten, für die
Tätigkeiten in Bibliotheken auf Hochschulniveau ausgebildet zu werden.
Sowohl in Zürich als auch in Bern gibt es MAS-Studiengänge, grundständig
bildet aber nur noch die Fachhochschule in Genf (Haute école de gestion
de Genève) aus, wobei diese nicht nur einen französischen, sondern auch
einen bilingualen Studiengang (französisch, deutsch) anbietet. Unterhalb
der Hochschulen existiert eine Berufsausbildung sowie ein Kurs von 124
Lektionen von je 45 Minuten, der von der Schweizerischen
Arbeitsgemeinschaft für allgemeine Öffentliche Bibliotheken akkreditiert
wird und vor allem für Schul- und Gemeindebibliotheken beliebt ist.
Angesichts dessen, dass die Sprachgrenzen in der Schweiz dem Anspruch
nach durchlässig, in der Praxis aber doch wirksam sind, stellt das
Studium an der HTW für die deutschsprachigen Bibliotheken - sowie
Archive und Dokumentationseinrichtungen - praktisch die einzige
Ausbildungseinrichtung dieser Art in der Schweiz dar.\footnote{Nicht
  berücksichtigt sind dabei verschiedene Angebote im tertiären
  Weiterbildungsbereich mit berufsbegleitenden Master-Studiengängen an
  der HTW Chur (MAS), der Universität und Zentralbibliothek Zürich
  (MAS), den Universitäten Lausanne und Bern (MALIS).}

Seit die beiden Autoren dieses Textes im Laufe des Jahres 2012 ihre
Unterrichtsaufgaben übernahmen, haben sie diesen in gemeinsamer
Absprache verändert und legen heute grossen Wert auf die Einbindung von
Forschungsmethoden in den Unterricht. Der vorliegende Text berichtet
über die Grundüberlegungen, welche hinter dieser Orientierung stehen und
stellt Vorstellungen zur zukünftigen Entwicklung des schweizerischen
Bibliothekswesens, auf die sich diese Orientierung bezieht, dar.
Anschliessend schildern die Autoren, wie sie Forschungsmethoden in die
Lehre in unterschiedliche Fächer integrieren und präsentieren einige
ausgewählte Abschlussarbeiten, auf die dieser Unterricht Einfluss hatte.
Zum Abschluss diskutieren die Autoren, inwieweit die neue Orientierung
ihre Zielsetzung erreicht hat und welche Schritte weiterhin notwendig
wären, um - was das Ziel der Neuorientierung war - die Absolventinnen
und Absolventen auf die zukünftigen Tätigkeiten in Bibliotheken besser
vorzubereiten und gleichzeitig die Bibliotheksszene in der Schweiz zu
bereichern.

\section*{Was heisst \enquote{Praxisorientierung} für die Lehre an
einer Fachhochschule und weshalb soll diese Forschungsmethoden
unterrichten?}\label{was-heisst-praxisorientierung-fuxfcr-die-lehre-an-einer-fachhochschule-und-weshalb-soll-diese-forschungsmethoden-unterrichten}

Fachhochschulen wurden in der Schweiz erst 1998 zugelassen und befinden
sich weiterhin in einer Entwicklungsphase. Beispielsweise wurde die
Akkreditierungsform erst Anfang 2015 wieder verändert. Die
Fachhochschulen haben in der schweizerischen Bildungslandschaft noch
nicht den starken Platz gefunden, den sie in der deutschen haben.
Grundidee ist und war es aber, Einrichtungen zu schaffen, deren Lehre
und Forschung sich mehr an praktischen Themen orientieren als in den
Universitäten, aber auf einem akademischen Niveau stattfinden und sich
damit von Höheren Fachschulen, die unterhalb der Fachhochschulen
angeordnet werden, abzugrenzen. Gleichzeitig haben Fachhochschulen den
Auftrag, vor allem solche Forschung zu betreiben, die praktisch
angewendet werden kann und für den Wissenstransfer der Ergebnisse zu
sorgen; obwohl gleichzeitig zum Beispiel der Schweizer Nationalfonds
betont, dass auch an Fachhochschulen Grundlagenforschung betrieben
werden könne.

Diese Situation führt selbstverständlich immer wieder zu Irritationen,
beispielsweise, wenn bei Forschenden an Fachhochschulen der Eindruck
entsteht, sowohl bei der Forschungsförderung durch den Schweizer
Nationalfonds als auch bei anderen Förderprogrammen grundsätzlich
strukturell benachteiligt und teilweise von Forschungsgeldern direkt
abgeschnitten zu werden. Stattdessen sind Fachhochschulen darauf
angewiesen, Drittmittel von anderen Stellen, hauptsächlich Firmen,
einzuwerben. Das für Fachhochschulen wichtigste Förderprogramm, das von
der KTI (Kommission für Technologie und Innovation) betrieben wird,
basiert denn auch auf der Zusammenarbeit mit Firmen und bezweckt primär
den Wissenstransfer.\footnote{Vergleiche \url{http://www.kti.admin.ch}}
Diese Beschränkung wird im Rahmen der Fachhochschulen immer wieder
beklagt. Andere Irritationen entstehen, wenn potentielle
Arbeitgeberinnen und Arbeitgeber zukünftiger Absolvierender, genauso wie
einige Studierende oder wissenschaftspolitische Akteure, nicht
wahrzunehmen scheinen, dass es sich bei den Fachhochschulen um
akademische Einrichtungen handelt und deshalb Anforderungen an die
Ausbildung stellen, die mit dem Selbstbild der Fachhochschulen nicht
übereinstimmen. Zudem ist das Selbstbild von Fachhochschulen in der
Schweiz wenig gefestigt. Gleichzeitig ermöglicht diese Situation aber
auch, dass sich Lehrende und Forschende recht aktiv in die weitere
Gestaltung von Fachhochschulen einbringen können.

Die beiden Autoren dieses Textes sind den akademisch orientierten
Dozierenden zuzuordnen. Ihrer Meinung nach ändern sich Bibliotheken als
Institutionen aktuell tiefgreifend; die Arbeit in Bibliotheken erfordert
immer mehr Personen, die sich nicht auf die gute Anwendung einiger
weniger spezialisierter Wissensbestände konzentrieren, sondern
eigenständig Entscheidungen treffen und sich regelmässig neues Wissen
erarbeiten. Gefragt ist eine immer grössere Eigenverantwortung des
Personals - mit RDA zum Beispiel auch im Bereich der Erschliessung -
sowie, analog zu anderen Bereichen der Gesellschaft, die sogenannten
Soft Skills wie Kommunikationsfähigkeit, Zeitmanagement,
Stressbewältigung und Empathie. Oder anders: Der Beobachtung der beiden
Autoren nach wird auch die Arbeit in Bibliotheken immer mehr zu einem
Handeln in Situationen mit unvollständigen Informationen und unter
ständigem Entscheidungsdruck. Diese Situation ist bekannt, wird aber oft
einigermassen ignoriert, insbesondere, wenn mit dem Begriff
\enquote{Lebenslanges Lernen} der Eindruck vermittelt wird, das alle
notwendigen Fähigkeiten und Wissensbestände im Laufe des Lebens einfach
in Weiterbildungen nachträglich erworben werden können.

Die Autoren haben als Lehrende eine Verantwortung für ihre Studierenden:
Sie wollen sie nicht mit veraltetem Wissen in das Leben nach der
Ausbildung entlassen, sondern ihnen die Fähigkeiten und Kompetenzen
mitgeben, die sie für die aktive Gestaltung ihres Lebens und
Berufsalltags benötigen.\footnote{Zwei Dinge sind anzumerken. Zum einen
  ist für einen Teil der Studierenden das Studium an der HTW nicht die
  erste Ausbildung. Eine grosse Zahl der Studierenden ist aus anderen
  Berufen in Bibliotheken gewechselt oder plant diesen Wechsel und nutzt
  das Studium als Aufstiegsweiterbildung. Dabei haben viele Studierende
  schon Berufe verlassen, die sie offenbar als monoton und ohne
  Zukunftsperspektive angesehen haben, beispielsweise eine erstaunliche
  grosse Anzahl von Buchhändlern und Buchhändlerinnen. Diesen
  Studierenden wieder einfach feststehende Wissensbestände zu
  vermitteln, nur auf Bibliotheken bezogen, würde ihren Interessen
  widersprechen. Zum anderen ist das Studium nicht einfach nur als
  Wissenszuwachs zu begreifen. Die Studierenden sind mindestens drei
  Jahre mit ihrem Studium beschäftigt, das heisst auch, dass sie sich in
  dieser Zeit verändern, ihre Persönlichkeit (neu) ausrichten. Obwohl
  dies bei den Diskussionen um Curriculareformen und dem
  Bachelor-/Mastersystem oft vergessen geht, ist jedes Studium
  selbstverständlich auch eine Passage im Leben. Die Studierenden am
  Anfang des Studiums sind andere Menschen als die am Ende des Studiums.
  Akzeptiert man dies, steigt die Verantwortung der akademischen
  Einrichtungen für die jeweiligen Studierenden. Die Autoren des Textes
  versuchen in ihrer Lehre auf diese Verantwortung zu reagieren.}
Sicherlich ist auch das spekulativ, dennoch gehen die Autoren davon aus,
dass es eine der Aufgaben einer Ausbildung sein muss, den Studierenden
zu ermöglichen, sich zu Persönlichkeiten zu bilden, die sich zutrauen,
in Situationen mit unvollständigem Wissen zu handeln, ihr Handeln und
die Gründe für dieses Handeln zu reflektieren, eigenständig neues Wissen
zu akkumulieren und zu produzieren. Bibliotheken sehen sich hier einer
dynamischen Entwicklung gegenüber, die ihre traditionelle Fokussierung
auf klare Regeln und strukturierte Abläufe grundlegend in Frage stellt.
Dies überträgt sich auf die Anforderungen an Mitarbeitende, was
zumindest von fortschrittlichen Bibliotheken verstanden und bei der
Neubesetzung von Stellen umgesetzt wird.

Neben der Auswahl der Themen, die in den Kursen der beiden Autoren
unterrichtet werden, wird versucht, diese Aufgabe mit vier Massnahmen
anzugehen: (1) Das eigenständige Arbeiten der Studierenden wird geübt,
indem sie nicht, wie in den meisten anderen Kursen, Klausuren schreiben
oder in Gruppen möglichst viele Aufgaben lösen, sondern als
Leistungsnachweise Hausarbeiten schreiben, die wissenschaftlichen
Standards entsprechen müssen. (2) Der Unterricht der beiden Autoren ist
geprägt vom freien Unterrichtsgespräch und ständigen Aufforderungen an
die Studierenden, Entscheidungen zu treffen und zu begründen. Zudem wird
immer wieder betont, welche Aufgaben auf die Studierenden bei einer
Arbeit in Bibliotheken zukommen werden. Ein grosser Teil der
Studierenden wird im Anschluss an das Studium auf festen oder
befristeten Projektstellen eingestellt und hat dort oft eine grosse
Verantwortung, sowohl in finanzieller als auch organisatorischer
Hinsicht. Dies soll sie nicht unvorbereitet treffen. (3) Mit einer
Einbindung wissenschaftlicher Methoden in den Unterricht. Dabei können
Methoden nicht losgelöst von Inhalten vermittelt werden, gleichzeitig
lassen sich Methoden nur nachvollziehen, wenn sich bei den Studierenden
eine gewisse Haltung - eben das methodische und strukturierte Handeln in
Situationen mit unvollständigen Informationen - einstellt. (4) Durch
Projektarbeiten wird das eigenverantwortliche Handeln, die Arbeit im
Team und der Einsatz einschlägiger Methoden gefördert.

Ziel der beiden Autoren ist es, Studierende zu Menschen auszubilden, die
im Rahmen von Bibliotheken selbständig und reflektiert handeln können
sowie sich in die bibliothekarischen Diskussionen einbringen. Das
bedeutet selbstverständlich, dass das Üben spezifischer Fähigkeiten, die
zum Beispiel in Bibliotheksschulen der Vorläufereinirichtung der Genfer
Fachhochschule im Mittelpunkt standen, in die eigentliche
Bibliothekspraxis verlagert werden muss. Für dieses Vorgehen erfährt die
Hochschule sowohl Zuspruch als auch Widerspruch von Seiten der
potentiellen Arbeitgeberinnen und Arbeitgeber und ist deshalb auf eigene
Entscheidungen angewiesen,\footnote{Was, in gewisser Weise ironisch, ein
  Handeln in einer Situation mit unvollständigem Wissen darstellt.
  Offenbar wird jede Entscheidung im Bezug auf das Studium von den
  potentiellen Arbeitgeberinnen und Arbeitsgebern sowohl begrüsst als
  auch abgelehnt; ob potentielle Studierende das überhaupt wahrnehmen
  oder ob das für ihre Studienentscheidung relevant ist, ist nicht
  bekannt. Insoweit liegt die Verantwortung für die Entscheidung bei den
  Dozierenden, da das Umfeld kein eindeutiges Bild von der notwendigen
  Weiterentwicklung zeichnet. Auch dies ist eine Situation, die
  vielleicht unbefriedigend ist, wenn die Fachhochschule als
  Dienstleister für die Wirtschaft - oder hier die Bibliotheken -
  angesehen wird; wie das Teile der schweizerischen Bildungspolitik zu
  tun scheinen. Aber es ist eine realistische Situation, welche den
  Unterrichtenden Möglichkeiten zur Gestaltung bietet.} wobei die
Autoren bislang auf die Unterstützung durch die Studienleitung und den
Fachbeirat des Instituts bauen können. Gerade die
Bibliotheksvertreterinnen und -vertreter im Fachbeirat haben in der
Diskussion um die aktuelle Curriculumsreform betont, dass von den
Absolventinnen und Absolventen in erster Linie erwartet wird, dass sie
flexibel, offen und neugierig sind und in den Bibliotheken als
\enquote{Innovationsträger} agieren können.

\section*{Was sollen wissenschaftliche Methoden in der Ausbildung
für Bibliothekswissenschaft an der HTW Chur
erreichen?}\label{was-sollen-wissenschaftliche-methoden-in-der-ausbildung-fuxfcr-bibliothekswissenschaft-an-der-htw-chur-erreichen}

Die beiden Autoren dieses Textes haben in ihren Vorlesungen und
Seminaren in den letzten Jahren damit begonnen, wissenschaftliche
Methoden zu unterrichten und gleichzeitig die Studierenden
wissenschaftlichen Texten auszusetzen. Dabei ist die Lehre an der HTW
Chur grundsätzlich relativ schulmässig organisiert. Die Studierenden
erhalten zu Beginn des Studiums eine Übersicht zu den Kursen, die in den
folgenden drei (Bachelor Vollzeit), vier (Bachelor Teilzeit) oder zwei
(Master) Jahren unterrichtet werden, inklusive einer Übersicht zu den
Inhalten dieser Kurse. Innerhalb dieses System ist nur eine kleine Zahl
von Projektkursen und Seminaren variabel. Die Studierenden erhalten die
Zusicherung, dass die Kurse im Laufe ihres Studiums erteilt werden. Dies
soll Planungssicherheit etablieren, gleichzeitig schränkt es die
Hochschule und die Dozierenden in ihrem Unterricht ein.\footnote{Abgesehen
  davon, dass nicht klar ist, wie in diesem System die Ergebnisse der
  Forschung in die Lehre einfliessen sollen. Dies kann nur sehr zufällig
  - wenn die Dozierenden eines Moduls zufällig an Projekten arbeiten,
  die genau zum Modul passen - oder sehr zeitverzögert nach einigen
  Jahren erfolgen. Sichtbar ist, dass die HTW, genauso wie viele andere
  schweizerische Fachhochschulen, aus einer Höheren Fachschule erwachsen
  ist und bislang offenbar noch nicht dem akademischen Ideal der Einheit
  von Forschung und Lehre folgt. Dass die Fachhochschulen in der Schweiz
  trotzdem zur Forschung angehalten werden und darauf Wert gelegt wird,
  dass die Forschenden auch unterrichten, ist einer der Widersprüche
  dieser Einrichtungen.} Insoweit können die beiden Autoren nicht ohne
Weiteres neue Fächer einführen oder solche, die ihnen überflüssig
erscheinen, streichen.\footnote{Dieses Problem stellt sich tatsächlich
  mit einem Fach, das von beiden Autoren unterrichtet wird und von
  dessen Sinnhaftigkeit die Dozierenden nicht überzeugt sind. Wenig
  überraschend wird dieses Fach auch durch die Studierenden regelmässig
  unterdurchschnittlich bewertet. Da aber dieses Fach im Studienplan
  ausgeschrieben ist und somit die Studierenden einen Anspruch auf eine
  Durchführung haben, kann es nicht einfach ersetzt werden. Mit der
  kommenden Curriculumsreform wird das Fach wegfallen, da sich die
  beiden Autoren dafür stark gemacht haben. Die starre Struktur des
  Studiengangs hemmt somit die Innovation in der Lehre, was aus Sicht
  der Autoren typisch für das stark verschulte Studium an Schweizer
  Fachhochschulen ist.} Erst in der aktuellen Curriculumsreform wird
dies möglich sein, wenn auch wieder nur für einen langen Zeitraum
gedacht.

Im Rahmen der Fächer die unterrichtet werden, haben die Dozierenden
allerdings einen grossen Gestaltungsspielraum, inklusive finanzieller
Mittel für Gastdozierende auch aus dem Ausland. Die beiden Autoren
unterrichten zur Zeit sowohl Kurse allein, aufgeteilt - das heisst je
nach Sitzung eine andere Person - und auch gemeinsam im Team. Unter
anderem unterrichten sie \enquote{Bibliotheksmanagement} (Mumenthaler),
\enquote{Bestandsmanagement} (Schuldt), \enquote{Grundlagen der
Informationswissenschaft}, Bibliothekswissenschaft (Mumenthaler,
Schuldt), \enquote{Sozialpsychologie und Benutzerberatung} (Mumenthaler,
Schuldt), \enquote{Aktuelle Trends in der Bibliothekswissenschaft und
-praxis} (Mumenthaler, Schuldt) sowie Seminare und Projektkurse. Der
Anspruch, wissenschaftliche Methoden im Unterricht zu vermitteln,
entstand bei den beiden Autoren zuerst ungeplant, wurde dann in
Diskussionen über die konkrete Unterrichtsgestaltung aber genauer
gefasst.

Im Rahmen der Gestaltungsmöglichkeiten bezüglich des Inhalts bestehender
Vorlesungen und Module können die Dozierenden einzelne Methoden in den
Unterricht integrieren. In Fächern wie \enquote{Sozialpsychologie und
Benutzerberatung} oder \enquote{Bestandsmanagement} werden gemeinsam mit
den Studierenden Studien aus wissenschaftlichen Zeitschriften gelesen
und auf deren Aussagekraft hin befragt. Die Studierenden sollen so einen
kritischen Blick auf Studien entwickeln und beispielsweise nach der
Sinnhaftigkeit von Fragestellungen und genutzten Methoden sowie der
Nachvollziehbarkeit von Ergebnisinterpretationen zu fragen lernen. Im
Modul \enquote{Sozialpsychologie und Benutzerberatung} besteht ein
Leistungsnachweis darin, in einer Informationseinrichtung die
Beobachtungsmethode Mystery Shopping durchzuführen. Dabei wird die
Methode im Präsenzunterricht vorgestellt und ein Beobachtungsbogen
erarbeitet. Die Umsetzung, also die eigentliche Beobachtung sowie die
Auswertung, erfolgt dann im Selbststudium. Der schriftliche Bericht
dient als Leistungsnachweis für den Teil Benutzerberatung in diesem
Modul.

Insbesondere im Fach \enquote{Aktuelle Trends in der
Bibliothekswissenschaft und -praxis} wird der Einsatz von
unterschiedlichen Methoden erprobt. Dies kann selbstverständlich immer
nur in einer reduzierten Form geschehen. In einer normalen
Unterrichtseinheit, welche an der HTW vier Mal 45 Minuten dauert, wird
jeweils eine Methode vorgestellt und mit ihren Vorteilen sowie Grenzen
diskutiert, an einem Beispiel zusammen mit den Studierenden
durchgespielt und anschliessend reflektiert. Wichtig ist dabei nicht,
dass die Durchführung den Kriterien wissenschaftlicher Arbeit
entspricht. Beispielsweise wird bei der Vorstellung qualitativer
Inhaltsanalysen nur ein erster, oberflächlicher Codierungsvorgang
durchgeführt. Wichtig ist vielmehr, dass die Studierenden die Methode
anwenden und erfahren, dass - wenn auch im kleinen Rahmen - durch diese
Anwendung neues Wissen entsteht und dass sie selber in der Lage sind,
dieses Wissen zu produzieren. In der abschliessenden Diskussion wird von
den Dozierenden immer darauf verwiesen, welcher Standard bei der
Durchführung der jeweiligen Methode existiert und wie die Ergebnisse in
Bibliotheken oder der Bibliothekspolitik verwendet werden könnten.
Grundsätzlich soll bei den Studierenden ein Interesse an der Nutzung
solcher Methoden sowie der kritischen Reflexion von Studien, die zum
Beispiel in bibliothekarischen Zeitschriften publiziert werden,
gefördert werden. Gleichzeitig soll dem \enquote{wissenschaftlichen
Wissen}, dass in solchen Studien dargestellt wird, der Mythos der
unangreifbaren Faktizität und der unverstehbaren Komplexität genommen
werden.

\paragraph{Vorteile für Studierende}\label{vorteile-fuxfcr-studierende}

Die Autoren des vorliegenden Textes erhoffen sich durch diese Einbindung
von Forschungsmethoden eine ganze Reihe von Vorteilen für die
Studierenden.

\begin{itemize}
\item
  Die Studierenden werden ihr Wissen darüber, welche Forschungsmethoden
  existieren, vor allem auf den Projektstellen in Bibliotheken oder
  ähnlichen Positionen nutzen können. Dabei sollten sie in der Lage
  sein, nicht nur die nächstliegenden Methoden auszuwählen, sondern
  wissen, wie breit die Möglichkeiten jeweils sind und selber in der
  Lage sein, nach Methoden zu recherchieren, ihren Einsatz zu planen und
  die Grenzen von Methoden zu erkennen. Eine Hoffnung ist auch, dass die
  Studierenden beginnen, auf die Ergebnisse schon einmal durchgeführter
  Studien in ähnlichen Zusammenhängen zurückzugreifen und diese nicht,
  wie dies oft vorkommt, regelmässig zu wiederholen.
\item
  In zahlreichen Disziplinen, in der Schweiz vor allem in
  Medizinberufen, hat sich eine evidenzbasierte berufliche Praxis
  etabliert. Entscheidungen werden auf der Grundlage von kleinen
  Forschungsprojekten und wissenschaflichem Wissen, das auf die
  jeweilige Situation bezogen wird, getroffen. Um dies zu ermöglichen,
  müssen die im jeweiligen Feld Tätigen in der Lage sein, mit solchem
  Wissen umzugehen, es zu recherchieren, zu interpretieren und in die
  eigene Praxis einzubauen. Während in der Schweiz Höhere Fachschulen
  für Medizinberufe sowie Pädagogische Hochschulen dazu übergegangen
  sind, grosse Teile des Unterrichts in Form von evidenzbasierten
  Projekten zu organisieren, steht ein solcher Wandel in Bibliotheken

  \begin{itemize}
  \itemsep1pt\parskip0pt\parsep0pt
  \item
    trotz längerer Diskussionen und Projekte unter dem Titel
    \enquote{Evidence Based Library Practice} vor allem in Kanada, auf
    die regelmässig verwiesen wird - aus. Die Studierenden sollen aber
    darauf vorbereitet werden, Prinzipien dieser evidenzbasierten
    Arbeitsweise nutzen zu können. Beide Autoren sehen eine solche
    Praxis, wenn sie reflektiert ist und nicht andere Formen der
    Wissensproduktion ersetzt, als sinnvoll an.
  \end{itemize}
\item
  Methodenwissen ermöglicht immer auch ein Denken auf einer vom
  konkreten Alltag abgelösten Abstraktionsebene. Nur wer in der Lage
  ist, vom Einzelfall in einer Bibliothek zu abstrahieren und sowohl
  Strukturen als auch Zusammenhänge zu analysieren, kann sinnvoll
  wissenschaftliche Methoden anwenden. Dies unterscheidet zum Beispiel
  Umfragen als reines Instrument, um Stimmungen und Meinungen
  aufzunehmen, von Umfragen, die zum Testen von Hypothesen und zum
  Erweitern von Wissensbeständen eingesetzt werden. In der Überzeugung
  der beiden Autoren dieses Textes impliziert der Aufbau von
  Methodenwissen immer auch einen Schritt hin zu wissenschaftlichem
  Denken. Ein solches abstrakteres Denken wird den Studierenden bei der
  Gestaltung ihres Lebens und bei Entscheidungen, die sie im Rahmen von
  Bibliotheken auf höherer Ebene zu treffen haben, von Nutzen sein.
  Insbesondere wenn es um zukunftsträchtige Entscheidungen geht, kann
  wissenschaftliches Denken besser diffuse Hoffnungen und
  Werbeversprechen erhellen und somit für die Institution sinnvolle
  Entscheidungen ermöglichen.
\item
  Studierende sollten, so die Hoffnung, durch den regelmässigen Kontakt
  zu wissenschaftlichen Texten, auch solchen, die theoretisch angelegt
  sind, und Methodendiskussionen ein Selbstbewusstsein gegenüber dem
  wissenschaftlichen Denken aufbauen. Im Berufsalltag sollen sie sich
  zutrauen, solche Methoden einzusetzen und nicht einfach auf subjektive
  Theoriebildungen zurückgreifen. Sie sollen ihre Arbeit als Teil des
  Fachdiskurses ansehen, dazu den Fachdiskurs aktiv verfolgen, sich
  darin verorten und auch über Publikationen zu ihm beitragen. Eine
  solche aktive Beteiligung wird nicht nur das Selbstvertrauen der dann
  New Professionalsstärken, sondern auch zu einem bibliothekarischen
  Diskurs führen, an dem sich, verglichen mit der jetzigen Situation,
  mehr Personen mit größerem Wissen und unterschiedlichen Meinungen
  beteiligen sowie die spätere Arbeit der Studierenden interessanter
  machen.
\end{itemize}

\paragraph{Vorteile für Bibliotheken in der
Schweiz}\label{vorteile-fuxfcr-bibliotheken-in-der-schweiz}

Grundsätzlich sollen, so das Verständnis in der schweizerischen
Bildungspolitik und bei einigen potentiellen Arbeitgeberinnen und
Arbeitgebern, die Fachhochschulen in der Schweiz genau das Personal
ausbilden, welches auf dem Arbeitsmarkt, im hier diskutierten Fall vor
allem in den Bibliotheken, benötigt wird. Dieses utilitaristische Denken
ist selbstverständlich unrealistisch: Konsens ist es heute, davon
auszugehen, dass sich die Arbeitswelt so rapide ändert, dass
Ausbildungsinhalte schnell veralten und dass deshalb in den
Ausbildungsgängen vor allem grundlegende Kompetenzen vermittelt werden
müssen, die Personen ermöglichen, sich an ständig neue Herausforderungen
und Arbeitsweisen anzupassen. Unbeachtet bleibt allerdings oft, dass
Personen, die solche Kompetenzen haben, nicht nur auf neue Anforderungen
reagieren, sondern selber ihre Umwelt und damit auch die Arbeitswelt
umgestalten. Nicht alle Veränderungen vollziehen sich, weil sie
notwendig sind; viele vollziehen sich auch, weil die Mitarbeiterinnen
und Mitarbeiter Prozesse ändern und verbessern wollen, weil sie
bestimmte neue Arbeitsweisen erwarten oder etablieren.

Diese Entwicklung dürfte sich auch in der Schweiz vollziehen. Angesichts
dessen, dass viele Absolvierende der HTW Chur relativ schnell
einflussreiche Positionen in Bibliotheken in der Schweiz besetzen, ist
davon auszugehen, dass das, was an der HTW unterrichtet wird, die
Bibliotheken in der Schweiz mit beeinflusst. Bezogen auf die Integration
von wissenschaftlichen Methoden in den Unterricht gehen die Autoren des
Textes von folgenden Tendenzen aus, die sich möglicherweise in den
Bibliotheken der Deutschschweiz dadurch ergeben könnten, dass die
Studierenden im Anschluss an das Studium mit bestimmten Kenntnissen und
Kompetenzen in den Bibliotheken angestellt werden.

\begin{itemize}
\item
  Dadurch, dass Forschungsmethoden bekannt sind, aber auch gewusst wird,
  dass es zahlreiche weitere Methoden zur Lösung von bestimmten Fragen
  gibt, werden im Arbeitsalltag auch mehr Methoden eingesetzt.
  Grundsätzlich könnten Probleme eher als wissenschaftlich anzugehende
  Fragestellungen begriffen und dann auch systematisch bearbeitet
  werden, als bisher, wo sie oft im Trial-and-Error-Verfahren oder mit
  relativ einfachen Methoden (Umfragen und Interviews) zu bearbeiten
  versucht werden. Dies wird zu anderen, hoffentlich besseren
  Erkenntnissen führen.
\item
  Wenn mehr wissenschaftliche Methoden in der Berufspraxis,
  beispielsweise bei der Projektarbeit, eingesetzt werden, werden die
  dafür nötigen Fähigkeiten und Kompetenzen bei allen Beteiligten
  gefördert, insbesondere, wenn dies reflektiert geschieht. Dies könnte
  zu einem Personal führen, das eher kritisch, fokussiert auf
  durchdachte und abgewogene Aussagen, auf strukturierte, aber
  ergebnisorientierte Arbeit ausgerichtet und stärker kommunikativ
  geprägt wäre als bisher.
\item
  Der vermehrte Einsatz von wissenschaftlichen Methoden zur Beantwortung
  von mehr Fragestellungen im Bibliotheksalltag würde die Tätigkeiten in
  den Bibliotheken abwechslungsreicher und interessanter machen. Damit
  einhergehen könnte auch das Zunehmen des Interesses von eher an
  abwechselnde Tätigkeiten begeisterten Personen an der Arbeit in
  Bibliotheken.
\item
  Wissenschaftliche Methoden in der Bibliothekspraxis würde die
  Erkenntnis der Bibliothek über sich selber erhöhen - schon allein, da
  Wissenschaft immer auch ein Mittel zur Selbstaufklärung darstellt

  \begin{itemize}
  \itemsep1pt\parskip0pt\parsep0pt
  \item
    und somit dazu beitragen, dass die bibliothekarische Arbeit
    qualitativ besser wird. Gleichzeitig würde so mehr Wissen produziert
    werden, das auch in den bibliothekswissenschaftlichen Diskurs
    einfliessen könnte.\footnote{Strukturell gesehen könnte damit die
      praxisorientierte Forschung, die bislang an den Fachhochschulen
      verortet wird, mehr in den Bibliotheken selber betrieben werden.
      Wäre eine ausreichende Forschungsfinanzierung gegeben, könnten die
      Fachhochschulen sich dann mit einer theoriegetriebenen Forschung
      befassen und somit den Wissensstand, auf dem praxisorientierte
      Forschung fussen soll, massiv erhöhen.}
  \end{itemize}
\item
  Letzlich kann dies zu einer - analog zu anderen Berufen -
  evidenzbasierten Bibliotheksarbeit führen. Damit würden Bibliotheken
  erfolgreicher die laufenden und neuen Herausforderungen bewältigen
  können.
\item
  Die Öffnung für wissenschaftliche Methoden impliziert auch eine
  Öffnung gegenüber internationalen Entwicklungen und Erkenntnissen. Zu
  Beginn einer wissenschaftlichen Untersuchung steht in der Regel eine
  Literaturanalyse, die sich zwangsläufig auch mit Studien und Berichten
  aus anderen Ländern befassen muss. Sich auf den wissenschaftlichen
  Diskurs einzulassen bedeutet auch, sich mit Entwicklungen in anderen
  Weltregionen zu befassen und Lehren aus den dortigen Erfahrungen zu
  ziehen.
\end{itemize}

\paragraph{Methoden in der Curriculumsreform
2015}\label{methoden-in-der-curriculumsreform-2015}

Konnten die beiden Autoren in den ersten Jahren ihrer Dozententätigkeit
punktuell Methoden in die bestehenden Unterrichtsmodule integrieren,
begann im Jahr 2014 eine grundlegende Curriculumsreform. Ziel der Reform
ist es, das gesamte Bachelorstudium der Informationswissenschaft an der
HTW Chur attraktiver zu gestalten und klarer zu positionieren. Es wird
anstelle der eher kleinen Vertiefungen vier deutlich umfangreichere
\enquote{Majors} geben, einer davon in Bibliotheksmanagement. Beim
Aufbau der Module wird nach den Vorgaben der Studienleitung dabei vom
Ansatz ausgegangen, dass zunächst die Kompetenzen formuliert werden,
welche die Studierenden im Modul erreichen sollen und über welche sie
bei Eintritt in ein weiteres Modul verfügen sollten. Es wird also viel
stärker von Fähigkeiten, Kompetenzen und Soft Skills ausgegangen als
bisher.

Innerhalb des Majors Bibliotheksmanagement werden zum Teil komplett neue
Module angeboten, wie zum Beispiel Bibliotheksdienstleistungen oder
Bibliotheksinformatik. Zum anderen wird die Möglichkeit geschaffen, ein
eigentliches Methodenseminar anzubieten, in dem künftig Methoden
vermittelt und ihr Einsatz angewendet werden können. Gleichzeitig werden
im Grundstudium bereits Vorlesungen zur empirischen Sozialforschung und
zu statistischer Analyse angeboten (sowohl heute als auch nach der
Curriculumsreform). Das geplante Methodenseminar wird auf diesen
Grundlagen aufbauen und zusätzliche Methoden behandeln, beispielsweise
ethnografische Methoden. Es ist angedacht, ausgehend von konkreten
Fragestellungen in Bibliotheken, gemeinsam mit den Studierenden mögliche
Untersuchungsdesigns zu entwickeln und zu diskutieren oder auch Methoden
konkret in kleineren Projekten einzusetzen. Ein besonderer Fokus wird
hier auf Beobachtungsmethoden gelegt, zum Beispiel Count-the-Traffic.

Die bisherigen Erfahrungen in studentischen Arbeiten (siehe unten) haben
gezeigt, dass diese Methoden im Studium nicht nur vermittelt, sondern
auch geübt werden sollten. Als Lernziel wird entsprechend formuliert,
dass die Studierenden in der Lage sein sollen, geeignete Methoden
auszuwählen und anzuwenden, die Kenntnis allein ist nicht ausreichend.
Neben dem Methodenseminar sind zwei Projektkurse und ein weiteres
Seminar, das majorübergreifend für alle Studierenden gelten wird,
geplant. In den Projektkursen erhalten die Studierenden die Gelegenheit,
das in der Vorlesung Projektmanagement (Grundstudium) erworbene Wissen
anzuwenden und in einer konkreten Projektsituation umzusetzen. Hier wird
der Schwerpunkt bei Projekten im Kontext von Um- oder Neubauten von
Bibliotheken oder von Neukonzeptionen gesetzt. Bisher wurden im Rahmen
der Vorlesung Bibliotheksmanagement begleitend solche Projekte
bearbeitet, die mit ihrem Praxisbezug bei den Studierenden beliebt,
jedoch auch sehr zeitaufwändig waren. Indem diese Projekte einen eigenen
Ort im Curriculum erhalten, kann hier der Aufwand grösser und
realitätsnaher sein. Das überarbeitete Curriculum soll zum
Herbstsemester 2015 in Kraft treten.

\section*{Beispiele für die Einbindung von wissenschaftlichen
Methoden in die Lehre an der HTW Chur und Ergebnisse dieser
Einbindung}\label{beispiele-fuxfcr-die-einbindung-von-wissenschaftlichen-methoden-in-die-lehre-an-der-htw-chur-und-ergebnisse-dieser-einbindung}

\paragraph{Beispiel 1: Seminar \enquote{Was tun Nutzerinnen und Nutzer
in der
Bibliothek?}}\label{beispiel-1-seminar-was-tun-nutzerinnen-und-nutzer-in-der-bibliothek}

Wenn es den Autoren möglich ist, ein Seminar oder einen Projektkurs
durchzuführen - diese werden immer in Konkurrenz zu anderen Dozierenden
beantragt und wechseln jährlich -, versuchen sie bereits bisher, dies
ebenfalls zur Vermittlung von wissenschaftlichen Methoden zu nutzen.
Herausragend war dafür ein Seminar, dass 2013 unter dem Titel
\enquote{Was tun Nutzerinnen und Nutzer in der Bibliothek?} durchgeführt
wurde und grundsätzlich als Methodenseminar geplant war.

In diesem Seminar erprobten die Studierenden die Nutzung verschiedener
Methoden, um die im Titel des Seminars enthaltene Frage beantworten zu
können. In vier Gruppen erarbeiteten sie - unter Zuhilfenahme von zuvor
zusammengestellten Texten - unterschiedliche Methodenstränge (Umfragen,
Online-Forschung, sozialwissenschaftliche und ethnologische Ansätze). In
jeder Sitzung stellte eine Gruppe die Methoden, die sie erarbeitet
hatten, vor, inklusive einem Ablaufplan für ein Forschungsprojekt mit
dieser Methode, den Vorteilen, Grenzen, forschungsethischen Problemen
und Beispielen, in denen diese Methoden in Bibliotheken eingesetzt
worden waren. Anschliessend wurden in einem Rollenspiel die restlichen
Studierenden als Bibliotheksleitung einer imaginären Bibliothek (in
jeder Stunde eine andere, so dass unterschiedliche Bibliotheksformen und
Herausforderungen bearbeitet werden konnten) vor die Aufgabe gestellt,
zusammen mit den Forschenden - also der jeweiligen Gruppe, die als
Expertinnen und Experten für \enquote{ihre} Methode galten - einen
Forschungsplan zur Frage, was Nutzerinnen und Nutzer in der Bibliothek
tun, zu entwerfen. Dabei wurden unter den Studierenden jeweils per Los
Positionen verteilt. Eine Person stellte zum Beispiel die
Finanzabteilung dar und hatte auf einen möglichst ausgeglichen Etat zu
achten, eine andere Person vertrat die Interessen des Personals und eine
weitere die Interessen der Nutzerinnen und Nutzer. Mithilfe der beiden
Autoren, die jeweils eine Rolle als imaginäre Unternehmsberater
spielten, wurden im Seminar vier potentielle Forschungspläne erarbeitet.
In der letzten Sitzung des Seminars wurden die Methoden und ihre
Anwendung, insbesondere eingebunden in den Alltag von Bibliotheken,
besprochen.

Sicherlich haben die Studierenden in diesem Seminar keine vollständige
Methodenausbildung erhalten. Vermittelt werden sollte aber, dass es
möglich ist, unterschiedliche wissenschaftliche Methoden in der
Bibliothekspraxis einzusetzen, sowie, dass dem Einsatz jeder
Forschungsmethode eine intensive Forschungsplanung vorausgehen muss. Die
Studierenden haben diese Punkte in der Abschlussevaluation als positive
Erkenntnisse benannt. Die beiden Autoren zogen zum Abschluss ebenfalls
ein positives Fazit. Sie wollten vor allem die Kritikfähigkeit der
Studierenden erhöhen und zugleich die Breite möglicher Methoden
aufzeigen.\footnote{Dabei wurde es im Rahmen des Seminars notwendig, den
  Studierenden zu untersagen, zu jeder Forschung und Forschungsfrage als
  Methode Umfragen vorzuschlagen. Offenbar tendieren nicht nur
  Bibliotheken, sondern auch Studierende dazu, diese Methode als
  naheliegendste anzusehen.} Ihrer Meinung nach war den Studierenden am
Ende des Seminars zudem bewusst, dass es sich lohnt, in konkreten Fällen
eigenständig nach Forschungsmethoden zu suchen.

Im Rahmen der Curriculumsreform wird, wie erwähnt, ein regelmässig
durchgeführtes Methodenseminar eingeführt. Es soll im Ablauf ähnlich
gestaltet sein. Dies gilt auch für ein im Frühjahrssemester 2015
geplantes Seminar zur Frage, ob Bibliotheken in der Schweiz als Dritter
Ort funktionieren. Dabei werden von der Fragestellung ausgehend
unterschiedliche methodische Ansätze evaluiert, um schliesslich eine
ausgewählte Methode anzuwenden und umzusetzen. Geplant ist im Rahmen
dieses Seminars auch, zumindest eine koordinierte Beobachtung über die
Art der Nutzung von Bibliotheksräumen durchzuführen.

\paragraph{Beispiel 2: Modul \enquote{Aktuelle Trends in
Bibliothekswissenschaft und
-praxis}}\label{beispiel-2-modul-aktuelle-trends-in-bibliothekswissenschaft-und--praxis}

Ziel diees Moduls, welches eingeführt wurde, bevor die beiden Autoren
zur HTW Chur wechselten, war es, einen Ort zu schaffen, an dem den
Studierenden flexibel die aktuellen Entwicklungen im Bibliotheksbereich
vermittelt werden könnten. Das Modul wird am Ende des Studiums besucht,
die Studierenden sollten daran anschliessend mit aktuellem Wissen in den
Arbeitsmarkt wechseln. Diese respektable Vorstellung hat in der Praxis
die Autoren als Lehrende nicht überzeugt. Im Gegensatz zu anderen Kursen
ermöglicht dieses Modul zwar eine gewisse Flexibilität in der
thematischen Gestaltung, was das Unterrichten interessanter macht.
Allerdings widerspricht der Aufbau dem Ziel, Studierende mit einem
zukunftsfähigen Wissen auszustatten, da der Eindruck entstehen könnte,
dass einen einmalige Vermittlung von Zuknuftsthemen für das gesamte
Berufsleben ausreichen würde. Ab dem Herbst 2012 wurde deshalb der
Hauptaugenmerk darauf gelegt, mit den Studierenden Strategien und
Methoden zu erarbeiten, die es ihnen ermöglichen, auch nach der
Ausbildung selbstständig den Trends in den bibliothekarischen und
bibliothekswissenschaftlichen Diskussionen zu folgen, diese sowohl für
eine Arbeit in Bibliotheken zu nutzen als auch kritisch zu hinterfragen
sowie selber an der genannten Diskussion teilzunehmen.

In gewisser Weise ist es das Ziel dieses Seminars, den Studierenden zu
zeigen, wie Trends in den Diskussionen entstehen, wie hilfreich, aber
auch prekär Trendberichte sein können und welche reichhaltigen Methoden
für die Analyse zukünftiger Aufgaben von Bibliotheken existieren. Auch
dies ist motiviert durch das Wissen, dass die Studierenden der HTW im
Anschluss an ihr Studium oft auf Stellen eingesetzt werden, bei denen
das Verfolgen von Trends und das Übersetzen dieser Trends auf die Ebene
einzelner Bibliotheken zum Aufgabenspektrum gehört.

Konkret werden im Modul mehrere, auf ein bis zwei Stunden Arbeitszeit
reduzierte, Methodenübungen durchgeführt und anschliessend reflektiert.
Dazu gehören Literaturanalysen, das Erarbeiten von
State-of-the-Art-Berichten, die Methodik des Horizon Reports - an dessen
Library Edition die HTW Chur aktiv beteiligt ist - als Beispiel für
Trendreports und die Szenariotechnik. Selbstverständlich sind die
Ergebnisse, die in diesem Modul erarbeitet werden, relativ unabgesichert
und oberflächlich. Dennoch scheinen die Studierenden, soweit dies zu
überprüfen ist, jeweils am Ende des Moduls sowohl das Selbstvertrauen,
solche Methoden anzuwenden und Aussagen aus ihnen zu generieren, als
auch einen kritischen Blick auf die Methoden erworben zu haben.

Was bisher nicht gelang, ist, Studierenden im Anschluss an das Studium
zur Publikation von Ergebnissen oder zur aktiven Teilnahme an
Konferenzen zu gewinnen, obwohl auch dies Ziel des Moduls ist und von
den Autoren dieses Textes regelmässig betont wird. So wird die Arbeit
von Redaktionen bibliothekarischer Zeitschriften sowohl aus der Sicht
von Schreibenden als auch von Redaktionsmitgliedern erläutert, es werden
je Durchführung des Moduls zwei Expertinnen und Experten eingeladen, die
nicht nur aktuelle Themen besprechen, sondern gleichzeitig zeigen
sollen, dass die Arbeit im Umfeld von Bibliotheken nicht nur in
klassischen bibliothekarischen Aufgaben bestehen kann. Findet des Modul
im Herbst statt, wird das seit drei Jahren an der HTW Chur durchgeführte
Infocamp - ein zweitägiges Barcamp mit einem hohen Anteil an
Impulsreferaten - ebenfalls in das Modul eingebunden. Die Studierenden
nehmen dann an dem Infocamp teil, dokumentieren die Diskussionen,
bereiten die Veranstaltung gemeinsam mit den Dozierenden nach und nutzen
ein von ihnen ausgewähltes, auf dem Infocamp aufgekommenes Thema für
eine Hausarbeit. Auch dies soll den Studierenden direkt die
Möglichkeiten, sich mittels Konferenzteilnahme in den bibliothekarischen
Diskurs einzubringen, aufzeigen.

\paragraph{Beispiel 3: Praxisprojekt im Modul
\enquote{Bibliotheksmanagement}}\label{beispiel-3-praxisprojekt-im-modul-bibliotheksmanagement}

Im Modul Bibliotheksmanagement werden verschiedene Managementmethoden
vermittelt, die im beruflichen Alltag - vor allem in leitender Funktion
- eingesetzt werden können: neben traditionelleren Themen wie
Personalmanagement oder strategischem Management auch neuere Aspekte wie
Innovations- oder Change Management. Es handelt sich dabei also nicht um
Forschungsmethoden im engeren Sinn. Ein wichtiges Element stellt jedoch
die Praxisarbeit dar, die ins Modul integriert ist und den
Leistungsnachweis darstellt. Dabei werden verschiedene Methoden
kombiniert, um die Neuausrichtung einer Bibliothek zu konzipieren.
Schwerpunkt bildet dabei das Raumkonzept, das an einem realen Beispiel
und auf der Grundlage eines wirklichen Bedarfs einer Bibliothek
entwickelt wird. Die Studierenden haben die Aufgabe - in der Regel in
Form einer Gruppenarbeit - die Bibliothek und ihre Bedürfnisse zu
analysieren und anschliessend eine oder mehrere Varianten für ein neues
Bibliothekskonzept zu entwickeln. Bei der Visualisierung der Entwürfe
erhalten sie Unterstützung durch einen Dozenten aus dem Fachbereich
Architektur an der HTW Chur. Zum Abschluss werden die Vorschläge vor den
Auftraggebern präsentiert. Durch die Arbeit an einem realen, konkreten
Fall können verschiedenste Methoden kombiniert und angewandt werden. Der
Lerneffekt ist sehr hoch - und die Praxisarbeit wird von den
Studierenden meist sehr geschätzt, auch wenn sie einen verhältnismässig
hohen Aufwand bedeutet. Im Rahmen der Curriculumsreform erhält das
Praxisprojekt ein eigenes Modul, so dass eine noch vertieftere
Auseinandersetzung mit dem Thema ermöglicht wird.

\paragraph{Beispiel 4: Mystery Shopping im Teilmodul
\enquote{Benutzerberatung}}\label{beispiel-4-mystery-shopping-im-teilmodul-benutzerberatung}

Im Modul \enquote{Sozialpsychologie und Benutzerberatung} wird die
Methode Mystery Shopping zur Qualitätssicherung von Dienstleistungen
eingeführt. Mystery Shopping geniesst zwar nicht unbedingt den Ruf einer
streng wissenschaftlichen Methode, doch ist sie als teilnehmende
Beobachtung durchaus ernstzunehmen. Im Unterricht wird das
Untersuchungsdesign gemeinsam entwickelt. Somit können sich die
Studierenden aktiv bei der Planung der Beobachtung (was wird
beobachtet?) sowie bei der Definition der Kriterien (wie wird
beobachtet? was wird bewertet?) einbringen. Da dieses Modul im
Grundstudium unterrichtet wird, machen die Studierenden hier zum ersten
Mal Erfahrungen mit einer eigenen Untersuchung. Die Ergebnisse sind
durchaus positiv: die Studierenden berichten, dass sich durch die
strukturierte Beobachtung des Auskunftsdienstes auch ihre eigene
Wahrnehmung verändert hat. Sie betrachten nun die eigene Tätigkeit
kritischer als zuvor - eine Rückmeldung aus den Reihen der
Teilzeitstudierenden, die bereits in einer Bibliothek arbeiteten. Andere
Studierende waren positiv überrascht von der Aufmerksamkeit, die ihrer
kleinen Forschungsarbeit von Seiten der untersuchten Bibliothek zuteil
wurde.

\paragraph{Weitere Beispiele: Ausgewählte Bachelor- und Masterarbeiten
in Bezug auf den Einsatz verschiedener
Methoden}\label{weitere-beispiele-ausgewuxe4hlte-bachelor--und-masterarbeiten-in-bezug-auf-den-einsatz-verschiedener-methoden}

Die Lehrenden an der HTW Chur haben einen, im Vergleich zu anderen
Hochschulen, relativ hohen Einfluss auf die studentischen
Abschlussarbeiten. Im Allgemeinen schlagen die Dozierenden Themen vor,
aus denen sich die Studierenden ein ihnen zusagendes aussuchen. Eigene
Vorschläge von Studierenden sind möglich, stellen aber Ausnahmen dar.
Nach der Themenwahl besprechen die Studierenden das Konzept ihrer
Arbeit, inklusive der Forschungsfragen und der gewählten
Forschungsmethoden mit den für ihre Arbeit zuständigen Dozierenden,
schreiben anschliessend auf dieser Basis ein Exposé, in dem sie Thema,
Fragestellung, Thesen, Methoden und Zeitplan für ihre Abschlussarbeit
darlegen. An dieses schliesst sich ein Kolloquium an, bei dem die
einzelnen Studierenden sowohl ihren Mitstudierenden als auch den
Dozierenden des Instituts ihr Vorhaben schildern und es im gemeinsamen
Gespräch gegebenenfalls anpassen. Erst daran anschliessend erarbeiten
und schreiben die Studierenden ihre Abschlussarbeit.\footnote{Dieses
  Vorgehen gilt vor allem für die Bachelorarbeiten. Vor allem der
  Berufsbegleitende MAS-Abschluss verzichtet auf das Kolloquium, dafür
  kann die Betreuung durch die Dozierenden enger sein.}

\subparagraph{Ausrichtung der Abschlussarbeiten auf
Forschungsmethoden}\label{ausrichtung-der-abschlussarbeiten-auf-forschungsmethoden}

Nachdem durch die beiden Autoren 2012 die ersten Abschlussarbeiten an
der HTW Chur betreut worden waren, wuchs bei ihnen ein gewisses
Unbehagen. Einerseits besteht in diesem System immer die Gefahr, dass
Dozierende zu sehr Einfluss auf die studentischen Arbeiten nehmen und
die Studierenden gerade die Eigenständigkeit, die sie mit der Arbeit
unter Beweis stellen sollen, nicht zeigen können. Andererseits
tendierten die Studierenden zu den immer gleichen Methoden, insbesondere
Umfragen und Expertinnen- und Experteninterviews. Dabei waren dies nicht
unbedingt die jeweils sinnvollsten Methoden, um die Forschungsfragen der
Studierenden zu untersuchen. Die Studierenden erhielten mit ihren
Arbeiten zwar Ergebnisse, aber nicht unbedingt immer die sinnvollsten
oder ausreichenden.

In gemeinsamer Absprache wurde versucht, dieses Problem in den folgenden
Abschlussarbeiten anzugehen. Dies ist auch möglich, da die beiden
Autoren kontinuierlich die meisten Bachelorarbeiten am gesamten Institut
betreuen. In den Ausschreibungen der Themen wurden seit 2013 nicht nur
die zu untersuchende Fragestellung vorgeschlagen, sondern explizite
Hinweise zu einer oder mehreren möglichen Forschungsmethoden gegeben.
Gleichzeitig wurden die Studierenden in den Gesprächen zu ihren
Abschlussarbeiten darauf hingewiesen, dass sie den Einsatz jeder Methode
in den Exposés begründen müssen, insbesondere dann, wenn sie Umfragen
oder Expertinnen- und Experteninterviews vornehmen wollten. Dabei wurden
sie explizit auf die reichhaltige Literatur zu Forschungsmethoden, auch
in anderen Fachgebieten, hingewiesen. Gerade bei jenen Studierenden, die
das entsprechende Methodenseminar bei den beiden Autoren besucht hatten,
stiessen die Vorschläge für die Anwendung neuartiger Methoden auf
Interesse.

Dieser an sich geringfügige Wechsel in der Betreuung der Arbeiten führte
relativ schnell dazu, dass die Bachelorarbeiten eine grössere Palette an
Methoden nutzten, dass diese Methoden in den Arbeiten relativ oft
diskutiert und ihr Einsatz sinnvoll geplant wurde. Eine Anzahl von
Arbeiten erhob sogar den Anspruch, bestimmte Methoden - insbesondere
strukturierte Beobachtungen wie \enquote{Count-the-Traffic} oder
\enquote{Sweeping the Floor} - fortzuschreiben. Offenbar sind die
Studierenden am Ende des Studiums in der Lage, Methoden sinnvoll auf
unterschiedliche Fragestellungen bezogen auszuwählen und selbstsicher
genug, diese Methoden nicht einfach nur anzuwenden, sondern auch zu
verändern.

Die klare Einbindung von Forschungsmethoden in den Unterricht wird von
den beiden Autoren erst seit einer kurzen Zeit vorgenommen. Erst Mitte
2015 werden die ersten Studierenden ihr Studium an der HTW beenden, die
während ihres gesamten Studiums mit diesem Ansatz unterrichtet
wurden.\footnote{Wobei es wichtig ist, noch einmal zu betonen, dass die
  beiden Autoren lediglich eine Anzahl von Modulen unterrichten. Andere
  Lehrende, welche die gleichen Studierenden unterrichten, setzen in
  ihrem Unterricht selbstverständlich andere Schwerpunkte.} Dennoch
scheint sich bereits jetzt in einer Anzahl von Abschlussarbeiten diese
Orientierung bemerkbar zu machen. Eine kurze Auswahl soll hier
vorgestellt werden, um die Potentiale eines solchen Unterrichts zu
verdeutlichen. Die Arbeiten nutzen alle unterschiedliche Methoden, die
sie zum Teil fortschreiben, um Wissen zu generieren, das für die
Bibliotheken neu und sinnvoll zu nutzen ist. Gleichzeitig zeigen sie
Potentiale des Einsatzes von ausgewählten Forschungsmethoden für
Bibliotheken auf. Im Rahmen der Forschungsstrategie im Schwerpunkt
Digitale Bibliotheken werden die Autoren systematisch studentische
Arbeiten als Vorstudien einbeziehen. Diese Arbeiten bilden dann einen
Teil in der Forschungsaktivität im Bereich Bibliothekswissenschaft.

\subparagraph{Erste Ergebnisse: Beispiele aus einigen
Abschlussarbeiten}\label{erste-ergebnisse-beispiele-aus-einigen-abschlussarbeiten}

Sibylle Schlumpf untersuchte das NutzerInnenverhalten in der Pestalozzi
Bibliothek Altstadt in Zürich mit Hilfe der Methode Count-the-Traffic.
Im Ergebnis wurde gezeigt, dass bestimmte Altersgruppen bestimmte Zonen
besonders nutzen (ältere Herren nutzen zum Beispiel die
Zeitschriftenecke intensiv) und dass andere Nutzergruppen die für sie
vorgesehenen Zonen nicht erwartungsgemäss in Anspruch nehmen - so
blieben die Jugendlichen der Jugendzone im Untergeschoss weitgehend
fern. Eine eingehende wissenschaftliche Untersuchung benötigte jedoch
mehr Zeit als für eine Bachelor-Thesis zur Verfügung steht.\footnote{Schlumpf,
  Sibylle (2013). Neue Raumkonzepte in Schweizer Öffentlichen
  Bibliotheken - Erfahrungen aus einem Bibliotheksumbau am Beispiel der
  Bibliothek Altstadt. Bachelorarbeit, HTW Chur.} In zwei
Bachelorarbeiten wurden 2014 die verwandte Methode des Sweeping the
Floor für Öffentliche und Wissenschaftliche Bibliotheken auf der Basis
empirischer Ergebnisse erweitert.\footnote{Jenni, Stephanie (2014).
  Zonierungskonzepte in Öffentlichen Bibliotheken: Anwendung und Nutzung
  in sechs Öffentlichen Bibliotheken in der Deutschschweiz.
  Bachelorarbeit, HTW Chur. Völker, Edith (2014). Arbeitszonen und
  Arbeitsräume in Wissenschaftlichen Bibliotheken: Entwurf einer Methode
  zur Beobachtung des Nutzerverhaltens in Wissenschaftlichen
  Bibliotheken. Bachelorarbeit, HTW Chur.} Mit einer Methodenkombination
von Sweeping the Floor-Methode und Interviews eruierte Sayako Bissig die
Nutzungsformen und Wahrnehmung eines bedienten und eines unbedienten
Bibliothekscafés. Im Ergebnis konnte sie zeigen, dass Bibliothekscafés
offenbar von den Nutzerinnen und Nutzern positiv als Öffnung der
Bibliothek wahrgenommen werden und auch dazu beitragen, dass
Bibliotheken flexibler genutzt werden, sie aber die Zahl der Nutzerinnen
und Nutzer einer Bibliothek nicht relevant verändern.\footnote{Bissig,
  Sayako (2014). Das Bibliothekscafé als dritter Ort:
  Praxis-Untersuchung über den Zusammenhang und die
  Beeinflussungsmöglichkeiten zwischen einer öffentlichen Bibliothek und
  einem Café in den Stadtbibliotheken Aarau und Baden. Bachelorarbeit,
  HTW Chur.}

Andrea Breu untersuchte mit einer teilnehmenden Beobachtung das
Verhalten von Familien in der St.~Galler Freihandbibliothek. Hier sollte
die Frage untersucht werden, ob erwachsene Begleitpersonen von Kindern
während ihres Besuchs auch die Angebote für Erwachsene nutzen. Als Fazit
konnte festgestellt werden, dass dies vor allem die Väter tun, wenn sie
ihre Kinder in die Bibliothek begleiten. Ansonsten lassen sich Bestände
für die Zielgruppen Kinder und Erwachsene auch gut trennen, so wie das
in St.~Gallen aktuell geplant ist.\footnote{Breu, Andrea (2014).
  Familien in der Bibliothek: Eine Fallanalyse der Freihandbibliothek
  St.~Gallen. Bachelorarbeit, HTW Chur.}

Die strukturierte Beobachtung wandte Felix Hüppi in seiner Master-Thesis
zum Nutzungsverhalten in wissenschaftlichen Bibliotheken an. Auch wenn
die Beobachtungssequenz relativ kurz war, zeitigte sie aufschlussreiche
Ergebnisse. Auffällig war, dass die in der Bibliothek lernenden oder
arbeitenden Studierenden sich ganz an den Rhythmus des Stundenplans
anpassten. Sie pausierten zeitgleich mit den Unterrichtszeiten. Für die
kurze Erholung verliessen sie jeweils die Bibliothek - vermutlich um
Mitstudierende zu treffen, die ebenfalls in der Pause waren. Aufgrund
dieser Beobachtung stellte Felix Hüppi die These auf, dass
Hochschulbibliotheken kaum als Dritter Ort genutzt werden und sie sich
eher darauf konzentrieren sollten, ein attraktiver Ort zum Arbeiten
(also ein zweiter Ort) zu sein. Diese These wiederum wurde von den
Autoren für das Methodenseminar im Frühjahrssemester 2015 aufgegriffen
(siehe oben).\footnote{Hüppi, Felix (2014). Lernraum Bibliothek: Theorie
  und Schweizer Praxis mit Umsetzungsbeispiel für die neue
  Campusbibliothek Muttenz der FHNW. Masterarbeit, HTW Chur.}

Lucas Althaus nutzte für seine Bachelor-Thesis die Methode der
qualitativen Statistik, um ausgehend von den Kennzahlen der Schweizer
Bibliotheksstatistik den Einfluss der Einführung eines Angebots von
E-Books in Öffentlichen Bibliotheken auf die physische Ausleihe von
Medien zu untersuchen. Die Ausgangslage präsentierte sich als
unvollständig, zumal sich bei weitem nicht alle Schweizer Kantone an der
Bibliotheksstatistik beteiligen. Ein Manko dieser Erhebung besteht zudem
darin, dass keine konsolidierten Zahlen zur Nutzung von E-Books erhoben
werden und dass auch die Zahl der Besucherinnen und Besucher einer
Bibliothek (nicht nur die Ausleihen oder die Zahl der eingeschriebenen
aktiven Nutzerinnen und Nutzer) nicht flächendeckend erfasst wird. Bei
den digitalen Plattformen für E-Books (de facto handelt es sich mit der
Onleihe um genau ein Produkt) lassen sich wiederum die Nutzungszahlen -
so weit sie überhaupt zur Verfügung gestellt werden - nicht für die
einzelnen Bibliotheken eruieren, sondern müssen auf der Ebene der
Verbünde oder Konsortien betrachtet werden. Auch wenn also die
Ausgangslage nicht optimal für eine solche Fragestellung war, konnte
Lucas Althaus als Trend klar zeigen, dass die Einführung eines
E-Book-Angebots keine direkten Auswirkungen auf die Ausleihzahlen der
jeweiligen Bibliothek hat.\footnote{Althaus, Lucas (2014). Der Einfluss
  ihrer Digitalen Bibliotheksportale auf die Öffentlichen Bibliotheken
  der Schweiz. Bachelorarbeit, HTW Chur.}

Mit Usability-Methoden, die sie auf die E-Books-Angebote von
Bibliotheken anwandte, befasste Farah Aeschlimann sich in ihrer
Bachelor-Thesis. Dabei ging es nicht darum, einzelne Angebote zu prüfen,
sondern ein Instrument für die Evaluierung der Usability von
Bibliothekswebsites, die am Schweizerischen Institut für
Informationswissenschaft entwickelt wurde (CHEVAL), auf die
E-Book-Angebote zu übertragen. Im Ergebnis wurden Kriterien erarbeitet,
mit denen dieser Selbsttest für Bibliotheken ergänzt werden
soll.\footnote{Aeschlimann, Farah (2014). Usability von E-Book-Angeboten
  wissenschaftlicher Bibliotheken - Entwicklung eines Kriterienkatalogs.
  Bachelorarbeit, HTW Chur.}

Ruth Süess wandte für ihre Master-Thesis die Methode der Szenariotechnik
an, um zu untersuchen, inwiefern sich diese für die strategische Planung
einer Hochschulbibliothek eignet. Sie kam zu einem durchaus positiven
Fazit, obschon sich herausstellte, dass die Methode, konsequent
eingesetzt, sehr aufwändig ist. Eine besondere Herausforderung bei der
Szenariotechnik ist es, die negativen Szenarien mit derselben Konsequenz
durchzuspielen wie die positiven. Entsprechend benannte die Autorin
diese Szenarien nicht als negative, sondern als herausfordernde.
Ansonsten besteht die Gefahr, dass unliebsame Vorstellungen (zum
Beispiel die Schliessung einer Bibliothek) als mögliche Ergebnisse
einfach ausgeblendet werden.\footnote{Süess, Ruth (2014).
  Szenario-Technik: Entwickeln von Zukunftsszenarien am Beispiel einer
  Universitätsbibliothek. Masterarbeit, HTW Chur.}

\section*{Fazit}\label{fazit}

Die Autoren dieses Textes binden seit einiger Zeit in ihrer Lehre im
Bereich Bibliothekswissenschaft an der HTW Chur die Vermittlung von
verschiedenen Forschungsmethoden ein. Diese Einbindung beschränkt sich
nicht auf das Vorstellen der Methoden, sondern beinhaltet zumeist das
Ausprobieren und die kritische Reflexion derselben.

Grundsätzlich ist die Integration solcher Methoden, den Erfahrungen der
Autoren nach, einfach und im Rahmen zahlreicher Themen sinnvoll möglich.
Ebenso sind Studierende in der Lage, die Grundüberlegungen
unterschiedlicher Methoden schnell zu erfassen, diese Methoden, wenn
auch in vereinfachter Form, direkt anzuwenden und ihre Grenzen sowie
Potentiale zu reflektieren. Dabei kommt der Lehre zu Gute, dass sich ein
Grossteil der Studierenden im Anschluss an ihre Ausbildung keiner
wissenschaftlichen Karriere zuwenden werden, sondern Positionen in
Bibliotheken, oft Projektstellen, besetzen werden. Auf diesen Stellen
können sie Methoden einigermassen offen einsetzen und sind nicht den
gleichen Restriktionen unterworfen, wie in wissenschaftlichen
Anwendungsfällen. Dennoch profitieren sie von der Schulung im kritischen
Denken, die mit der Methodenlehre einhergeht, und der Einführung in das
wissenschaftliche Denken, insbesondere dem Selbstvertrauen, in
Situationen mit unzureichenden Informationen Entscheidungen zu treffen,
Thesen und Aussagen zu formulieren und zu handeln.

Die Motiviation der Studierenden, aber auch die Qualität und
Aussagekraft der Abschlussarbeiten scheint sich durch dieses Vorgehen
erhöht zu haben. Gleichzeitig fühlen sich auch die beiden Autoren durch
den geschilderten offenen Methodenunterricht herausgefordert und
motiviert. Dieser Unterricht bedeutet für die Autoren ebenfalls, sich
selbst ständig mit Methoden und Methodenentwicklungen
auseinanderzusetzen, mit den Studierenden zusammen, trotz thematischer
Wiederholungen, immer wieder neue Ergebnisse zu erarbeiten und offene
Diskussionen über die einzelnen Methoden zu führen. Dabei ist auch das
praktizierte Teamteaching der beiden Autoren von Vorteil, da sie so in
Diskussionen ihre manchmal gegenteiligen Meinungen darlegen können. Dies
soll den Studierenden auch verdeutlichen, dass alle Methoden und alle
Ergebnisse im bibliothekarischen Diskurs zur Diskussion stehen.
Grundsätzlich ist also, bei allen Einschränkungen, ein positives Fazit
zu ziehen.

Offen bleibt, ob es gelingen wird, die Studierenden auf lange Sicht zu
einer Beteiligung am bibliothekarischen Diskurs zu motivieren. Dies ist
zur Zeit nicht der Fall, auch wenn an der HTW Chur eine erstaunliche
Anzahl von Publikationsorganen verankert ist - im Schweizerischen
Institut für Informationswissenschaft arbeiten Redakteure der LIBREAS.
Library Ideas (einer der Autoren des Textes), der Informationspraxis
(der andere Autor des Textes), der B.I.T. online und des Beirats der
Medien \& Alter, zudem bestehen personell kurze Wege zum schweizerischen
Fachorgan Arbido (mehrere ehemalige Mitarbeitende) - auf die im
Unterricht des Öfteren verwiesen wird. Eine weitere
Publikationsmöglichkeit bietet das Blog des Instituts,\footnote{Vergleiche
  \url{http://blog.informationswissenschaft.ch}.} die durchaus häufiger
genutzt werden könnte. Eventuell muss dazu das Schreiben von Artikeln
und Referaten als Mittel zur Beteiligung an Diskussionen explizit in der
Lehre verankert werden.

Ebenso müssen die Studierenden auch die Möglichkeit erhalten, ihre
Fähigkeiten im Berufsalltag einzubringen. Dazu bedarf es noch einer
Öffnung seitens der schweizerischen Bibliotheken hin zu einer durch
wissenschaftliche Methoden untermauerte Berufspraxis. Auchist es
überdies die Aufgabe der HTW Chur, diesen Einrichtungen bekannt zu
machen, über welche Kompetenzen die Studierenden am Ende des Studiums
verfügen und wie deren Fähigkeiten dazu beitragen, dass Bibliotheken
stets am Puls der Zeit agieren und sich so in die verändernde
Gesellschaft integrieren können.

Nicht zuletzt verbleibt das, was in der Lehre an Methodenwissen
vermittelt werden kann, auf einer allgemeinen Ebene. Letztlich haben die
Autoren die Hoffnung, dass die Studierenden ermutigt werden, nach dem
Studium den Einsatz verschiedener Methoden zu erproben. An der HTW Chur
sollen sie dazu das Grundwissen beziehungsweise übergreifende
Kompetenzen erwerben. Angesichts der hohen Arbeitslast der Studierenden
während des Studiums ist dies nicht anders möglich.

Insoweit ist die Einbindung von Forschungsmethoden in die Lehre kein
Ersatz für eine aktive Methoden- und Theorieentwicklung. Diese muss
anderswo stattfinden, die Lehre kann nur auf deren Ergebnisse
zurückgreifen.

%autor
\begin{center}\rule{0.5\linewidth}{\linethickness}\end{center}

\textbf{Rudolf Mumenthaler} (Prof.) ist Dozent am Schweizerischen
Institut für Informationswissenschaft, HTW Chur, Vorstandsmitglied des
BIS (Bibliothek Information Schweiz) und Redakteur der
Informationspraxis. Zuvor war er tätig an der ETH-Bibliothek Zürich.

\textbf{Karsten Schuldt} (Dr.) ist Wissenschaftlicher Mitarbeiter am
Schweizerischen Institut für Informationswissenschaft, HTW Chur,
Lehrbeauftragter an der FH Potsdam und Redakteur der LIBREAS. Library
Ideas. Zuvor tätig am Interdisziplinären Zentrum für Bildungsforschung,
HU Berlin.

\end{document}
