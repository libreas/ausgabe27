\documentclass[a4paper,
fontsize=11pt,
%headings=small,
oneside,
numbers=noperiodatend,
parskip=half-,
bibliography=totoc,
final
]{scrartcl}

\usepackage{synttree}
\usepackage{graphicx}
\setkeys{Gin}{width=.4\textwidth} %default pics size

\graphicspath{{./plots/}}
\usepackage[english]{babel}
\usepackage[T1]{fontenc}
%\usepackage{amsmath}
\usepackage[utf8x]{inputenc}
\usepackage [hyphens]{url}
\usepackage{booktabs} 
\usepackage[left=2.4cm,right=2.4cm,top=2.3cm,bottom=2cm,includeheadfoot]{geometry}
\usepackage{eurosym}
\usepackage{multirow}
\usepackage[english]{varioref}
\setcapindent{1em}
\renewcommand{\labelitemi}{--}
\usepackage{paralist}
\usepackage{pdfpages}
\usepackage{lscape}
\usepackage{float}
\usepackage{acronym}
\usepackage{eurosym}
\usepackage[babel]{csquotes}
\usepackage{longtable,lscape}
\usepackage{mathpazo}
\usepackage[flushmargin,ragged]{footmisc} % left align footnote

\usepackage{listings}

\urlstyle{same}  % don't use monospace font for urls

\usepackage[fleqn]{amsmath}

%adjust fontsize for part

\usepackage{sectsty}
\partfont{\large}

%Das BibTeX-Zeichen mit \BibTeX setzen:
\def\symbol#1{\char #1\relax}
\def\bsl{{\tt\symbol{'134}}}
\def\BibTeX{{\rm B\kern-.05em{\sc i\kern-.025em b}\kern-.08em
    T\kern-.1667em\lower.7ex\hbox{E}\kern-.125emX}}

\usepackage{fancyhdr}
\fancyhf{}
\pagestyle{fancyplain}
\fancyhead[R]{\thepage}

%meta
%meta

\fancyhead[L]{A. Peters \\ %author
LIBREAS. Library Ideas, 27 (2015). % journal, issue, volume.
\href{http://nbn-resolving.de/urn:nbn:de:kobv:11-100229865
}{urn:nbn:de:kobv:11-100229865}} % urn
\fancyhead[R]{\thepage} %page number
\fancyfoot[L] {\textit{Creative Commons BY 3.0}} %licence
\fancyfoot[R] {\textit{ISSN: 1860-7950}}

\title{\LARGE{A Textbook for Infinite Learning. It’s a Small World, After All – Review of Sandra Hirsh [Ed.] (2015): Information Services Today: An Introduction. Lanham, MD: Rowman \& Littlefield. 514 p., ISBN 978-1-4422-3957-9}} %title %title
\author{Alison Peters} %author

\setcounter{page}{65}

\usepackage[colorlinks, linkcolor=black,citecolor=black, urlcolor=blue,
breaklinks= true]{hyperref}

\date{}
\begin{document}

\maketitle
\thispagestyle{fancyplain} 

%abstracts

%body
The first thing I took note of upon starting a master's degree in
library and information science (LIS) is that LIS professionals use and
must be knowledgeable of many academic fields -- and an infinite number
of acronyms and abbreviations. Thankfully, the new LIS textbook,
\emph{Information Services Today: An Introduction}, is well aware of
that fact, and of how confusing all the abbreviations can be for a new
student, or even a seasoned information professional. ASRS? ISIS? ILFA?
CCBY?\footnote{ASRS: automated storage and retrieval system. ISIS:
  International Society for Information Studies. ILFA: International
  Federation of Library Associations and Institutions. CC BY: Creative
  Commons Attribution Only} Don't worry, \emph{Information Services
Today} has got you covered with four pages worth of explanations. It's a
necessary part of understanding our profession, and just one small
example of the myriad of thoughtful details in a textbook compiled and
edited by Sandra Hirsh, an all-around intellectual
trainer-teacher-learner.

Today's textbooks must necessarily not be the same old hard-bound tomes
that were the standard. In this digital age, Hirsh, Director of San José
State University's iSchool, a completely 100 \% virtual educational
environment, intimately understood that an introductory textbook must
encapsulate the history of LIS, provide an overview of information
relevant to the field, and then take it one step further by making use
of the multimedia and collaborative tools at our disposal. As a benefit
of working with instructors and students from across the globe, all
connecting via online classrooms, document-sharing systems and
collaboration applications, the one thing a textbook compiled and edited
by Hirsh \emph{would not be} is stagnant. \enquote{Given how quickly our
field is changing,} Hirsh says, \enquote{it will be important to
continually update the material in this book and refresh the
perspectives.} So, working with publishers Rowman \& Littlefield, the
author created a textbook that would meet the challenges and unique
situations today's information professionals address on a daily basis.

\subsubsection{A Multipurpose Book for Many
Audiences}\label{a-multipurpose-book-for-many-audiences}

The book has four different audiences who will benefit from the content
in different ways: instructors, students, current information
professionals, and non-LIS professional laypeople.

Instructors can use this book to help their students learn all about the
field. What it means to be a library and information professional today,
with particular emphasis on how information organizations will remain
valuable entities in their communities, how they will continue to thrive
but need to remain creative, innovative, and technologically advanced.
It's a good lesson that applies to all of us.

The new student will learn the foundational core of the field,
understand where the profession is today, and explore career development
strategies useful when applying for and achieving the dream LIS job,
post-degree.

Current information professionals can use the book to refresh their
knowledge and learn about topics that might not have been covered when
they were in school. The entire \emph{Part V -- Information
Organizations: Management Skills for the Information Professional},
covers management of employees, collections, budgets -- everything you
need to be aware of when in a management role. The old expectation was
that employers would be responsible for ongoing skills and learning.
Today, each individual is responsible for taking charge of ongoing
learning, to keep up with the ever-changing knowledge base. That's where
the book comes in especially helpful.

And \emph{Information Services Today} is perfect for non-LIS
professionals who are curious about what our field is and what we do.
When friends, family, random strangers fallback on the old stereotypes
and assume being an LIS professional or a librarian is just about the
books, or ask, \enquote{What do librarians DO?}: buy or loan them a copy
of this book. In a nutshell, this book provides any reader with a
glimpse of what libraries, information organizations, and information
professionals \enquote{do} and provides an overall understanding of the
complex, technological, and global information environment we all live
in.

\subsubsection{For Your Information}\label{for-your-information}

But no matter where you are in the LIS-sphere, the best resources, those
that will be updated as information changes, are the book's supplemental
materials. That includes the usual discussion questions and informative
call-outs, but also sections titled \enquote{Check this Out},
hyperlinked highlights of examples and outside sources for further
study. And launching in April 2015 are webinars featuring authors from
the book
(\href{http://lj.libraryjournal.com/webcasts/hirshondemand}{\emph{http://lj.libraryjournal.com/webcasts/hirshondemand}})
discussing the trends and issues in their respective areas. Hosted by
\href{http://lj.libraryjournal.com}{\emph{Library Journal}}, each
50-minute nugget, moderated by Hirsh, offers an introduction of the
topic, presentations from panelists, and a brief Q\&A session where
panelists discuss key trends, competencies, and strategies for success
within the field of library and information service. It's all carefully
formulated to give readers the opportunity to go beyond the content on
the page and gain a deeper understanding of the topics.

So if you're considering makerspaces for your own library, Chapter 19:
\emph{Creation Culture and Makerspaces}, written by Kristin Fontichiaro,
\emph{won't} provide a DIY blueprint, but you will get an overview of
the history of these spaces, sometimes referred to as digital labs or
production studios. You will learn how they have developed in libraries
to support a community's \enquote{intellectual and personal interests};
how you can use them to support community inclusion; and, highly useful
if you are in charge of library programs or grant writing, understand
key questions to ask your library team before delving into a makerspace
plan. Then jump to the online supplements and go on a \emph{Makerspace
Virtual Tour} to visit a working space, in action. Follow up with
Chapter 36, where Barbara M. Jones discusses intellectual freedom, and
offers cautionary advice about the spaces, which, \enquote{like many
innovative information programs \ldots{} often lack intellectual freedom
policies.} Now you have an understanding of makerspaces, examples of how
they are used in libraries, and questions to consider when planning your
own space.

The LIS-specific abbreviations spelled out in \emph{Information Services
Today} are just one facet of our exhilarating
\enquote{meta-discipline}\footnote{Bates, M.J. (2007). \enquote{Defining
  the information disciplines in encyclopedia development}
  \emph{Information Research}, 12(4) paper colis29. Retrieved from:
  \url{http://InformationR.net/ir/12-4/colis/colis29.html}}, one that
encompasses academics from arts to humanities, science and logic. It is
clear that LIS instructors, students and professionals need to have a
general understanding of many things, with resources available to point
in the direction of further study, when necessary. Like the discipline
itself, \emph{Information Services Today} brings together many different
voices through its contributors, covering a multitude of topics all
combining to provide diverse perspectives and unique points of view.
Bottom line: this textbook is simply a good read. Contributing author
Paul Signorelli notes in Chapter 20: \emph{Infinite Learning}, that
\enquote{learners in many parts of the world have come to expect that
learning will occur when and where they need it.} \emph{Information
Services Today} is a book created expressly for lifelong learning, for
immediate access and thoughtful advice from experienced professionals
who have followed this path, and now share their expertise for the
benefit of the next generation. When your field is constantly changing,
life-long learning is a necessary path to education, and
\emph{Information Services Today} can be a key investment to LIS
enlightenment.

%autor
\begin{center}\rule{0.5\linewidth}{\linethickness}\end{center}

\textbf{Alison Peters} is currently obtaining her MLIS from
\href{http://ischool.sjsu.edu/}{\emph{San Jose State University's
iSchool}}; earned a B.A. in English from
\href{http://berkeley.edu/}{\emph{UC Berkeley}}; and, when not working,
querying, or in class, puts her M.F.A. from
\href{http://www.mills.edu/}{\emph{Mills College}} to good use and
shares her love for all things bookish on
\href{http://bookriot.com/}{\emph{Book Riot}}. She was inspired to write
for LIBREAS by the memory of her grandfather, a self-taught
German-speaking African American who, the story goes, translated for
Albert Einstein at a \href{http://www.caltech.edu/}{\emph{California
Institute of Technology (Cal Tech)}} lecture in 1931. You can find her
serious professional side on LinkedIn.

\end{document}