\documentclass[a4paper,
fontsize=11pt,
%headings=small,
oneside,
numbers=noperiodatend,
parskip=half-,
bibliography=totoc,
final
]{scrartcl}

\usepackage{synttree}
\usepackage{graphicx}
\setkeys{Gin}{width=.4\textwidth} %default pics size

\graphicspath{{./plots/}}
\usepackage[ngerman]{babel}
\usepackage[T1]{fontenc}
%\usepackage{amsmath}
\usepackage[utf8x]{inputenc}
\usepackage [hyphens]{url}
\usepackage{booktabs} 
\usepackage[left=2.4cm,right=2.4cm,top=2.3cm,bottom=2cm,includeheadfoot]{geometry}
\usepackage{eurosym}
\usepackage{multirow}
\usepackage[ngerman]{varioref}
\setcapindent{1em}
\renewcommand{\labelitemi}{--}
\usepackage{paralist}
\usepackage{pdfpages}
\usepackage{lscape}
\usepackage{float}
\usepackage{acronym}
\usepackage{eurosym}
\usepackage[babel]{csquotes}
\usepackage{longtable,lscape}
\usepackage{mathpazo}
\usepackage[flushmargin,ragged]{footmisc} % left align footnote

\usepackage{listings}

\urlstyle{same}  % don't use monospace font for urls

\usepackage[fleqn]{amsmath}

%adjust fontsize for part

\usepackage{sectsty}
\partfont{\large}

%Das BibTeX-Zeichen mit \BibTeX setzen:
\def\symbol#1{\char #1\relax}
\def\bsl{{\tt\symbol{'134}}}
\def\BibTeX{{\rm B\kern-.05em{\sc i\kern-.025em b}\kern-.08em
    T\kern-.1667em\lower.7ex\hbox{E}\kern-.125emX}}

\usepackage{fancyhdr}
\fancyhf{}
\pagestyle{fancyplain}
\fancyhead[R]{\thepage}

%meta
%meta

\fancyhead[L]{J. Waack \\ %author
LIBREAS. Library Ideas, 27 (2015). % journal, issue, volume.
\href{http://nbn-resolving.de/urn:nbn:de:kobv:11-100229865
}{urn:nbn:de:kobv:11-100229865}} % urn
\fancyhead[R]{\thepage} %page number
\fancyfoot[L] {\textit{Creative Commons BY 3.0}} %licence
\fancyfoot[R] {\textit{ISSN: 1860-7950}}

\title{\LARGE{Arctic Monkeys’ \emph{Library Pictures}}} %title %title
\author{Juliane Waack} %author

\setcounter{page}{68}

\usepackage[colorlinks, linkcolor=black,citecolor=black, urlcolor=blue,
breaklinks= true]{hyperref}

\date{}
\begin{document}

\maketitle
\thispagestyle{fancyplain} 

%abstracts

%body
Die Symbolik der Bibliothek steht in der akademischen als auch
literarischen Tradition meistens im Kontext der Wissensvermittlung, der
geheimen Schätze und im schlimmsten Fall der verstaubten Abstinenz vom
Leben. Doch eins, zwei Schritte weiter, sowohl in Literatur, Popmusik
als auch dem wahren Leben, offenbart sie sich als äußerst erotische
Umgebung, deren Ruhe nur zur Unterstreichung des heißen Atems, die
trockenen Fachbücher zur Kulisse der errötenden Haut dienen.

\enquote{Library Pictures}, säuselt Alex Turner dementsprechend im
Schlafzimmerton auf dem vierten Album (\enquote{Suck it and See}) der
Arctic Monkeys, ehemals halbstarke Britpopper, mittlerweile überlegte
Stonerocker, denen die Eleganz der Sechziger im Gitarrenanschlag liegt.

Sieht man sich den Text genauer an, so scheint das Aneinanderreihen von
Bildern keinerlei Hinweis darauf zu geben, was genau Turner uns hier
erzählen will, doch glaubt man dem ein oder anderen Hobby-Interpreten,
so geht es hier um den \enquote{slippery} Geschlechtsakt, der den
\enquote{library pictures of a quickening canoe} ähnelt.

Man möchte gar nicht widersprechen, zu sehr ähnelt der Rhythmus,
spielerisch und fast schon zurückhaltend zu Beginn und dann mit jeder
Sekunde schneller, aufgeregter, chaotischer, den Phasen des Liebemachens
(bis hin zur kakophonischen Klimax).

Dass es sich beim Sex in der Bibliothek sicher nicht um etwas
Alltägliches handelt - auch wenn die ein oder andere
Schlagzeile\footnote{\url{http://www.huffingtonpost.com/2015/04/10/library-girl-kendra-sunderland-weird-cam-requests_n_7043306.html}}
masturbierender Bibliotheksbesucherinnen etwas anderes suggerieren -
lesen wir aus Turners zweiter Songzeile \enquote{The first of its kind
to get to the moon}. Vielleicht ist sie auch deshalb so anziehend, die
Bibliothek, denn während es genug Umgebungen gibt, die das Hingeben der
Lust offensichtlich begünstigen (möge jeder hier seinen eigenen Ort des
Vergnügens einfügen), so wirkt die Bibliothek wie die Antithese der
sinnlichen Ekstase, immerhin herrscht hier der Geist, das Wort und
darüber hinaus auch noch ein Archivierungssystem, das jegliche
Spontanität ausschließt.

Doch gerade diese Ordnung scheint sie hervorzurufen, die schmutzigen
Gedanken, denn ähnlich Paul Austers Hauptcharakter im Roman
\enquote{Unsichtbar}\footnote{Auster, Paul \enquote{Unsichtbar} (2009),
  Rowohlt}, der vor lauter Langeweile als Bibliothekshiwi zum
regelmäßigen Masturbieren übergeht, ist wohl gerade die völlige
Abwesenheit der Körperlichkeit im stereotypisch aufgeräumten
Bibliotheksraum anziehend, da sie ja fast schon als Aufforderung zur
Belebung angesehen werden kann.

Und ganz verschwitzt und mit geröteten Wangen sorgt man für etwas
unsortierte Fleischlichkeit, die ganz selbstvergessen in einer Ecke
aufatmet. \enquote{You look as if you've all forgotten where you've
been.}

Songlink:
\href{https://www.youtube.com/watch?v=oU6hQaxSWF8}{\emph{https://www.youtube.com/watch?v=oU6hQaxSWF8}}

%autor
\begin{center}\rule{0.5\linewidth}{\linethickness}\end{center}

\textbf{Juliane Waack} wurde 1984 in Rostock geboren und studierte nur
knapp 19 Jahre später eben dort Anglistik/Amerikanistik, Kommunikation
und Philosophie an der Universität Rostock. Mittlerweile ist sie
technische Redakteurin, hat ihre eigene Sendung beim Rostocker
Lokalradio LOHRO (\url{http://www.lohro.de}) und bloggt mal mehr mal
weniger regelmäßig über Musik und mehr auf
\url{http://fichtenstein.wordpress.com} (englisch) und
\url{http://sickcellmate.wordpress.com} (deutsch).

\end{document}